\chapter{アセンブリ言語による記述の利点}
\section{目的}
\ref{chap1}章から\ref{chap4}章で得られた事実や考察に基づいて,アセンブリ言語で整列アルゴリズムを直接記述することに利点があるかどうかを明らかにする.
\section{考察}
\ref{chap4}章の実験結果より,高級言語である{\ttfamily Java}よりもアセンブリ言語で記述した選択アルゴリズムの方が,実行時間は短いことが分かっている.
ただ,先にも述べたように{\ttfamily Java}がコンパイラ・インタプリタ方式採用していることが大きな原因と考えられ,仮にコンパイラ方式を採用している言語で実験したとしても,一般に高級言語よりも低級言語の方が実行時間は短いとは断定できない.今回の実験に限ってはアセンブリ言語での記述によって整列アルゴリズムがより高速に実行できたので,この点についてはアセンブリ言語で記述する利点であろう.\par
アセンブリ言語はCPU内のレジスタの値の操作,主記憶領域の値の操作を1つずつ記入する.\ref{cha2}の実験を踏まえると,コード量は一般の高級言語に比べるととても多くなり複雑になる.
これは,プログラマーの人的ミスを引き起こしやすい要因でもあり,コードが複雑化するアセンブリ言語の欠点とも言える.
