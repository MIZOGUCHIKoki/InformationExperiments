\documentclass[a4paper,11pt]{ltjsreport}
\usepackage{amsmath,amssymb,array}
\usepackage[top=20truemm,bottom=20truemm,left=15truemm,right=15truemm]{geometry}
\usepackage{graphicx,color}
\usepackage{tikz,wrapfig,float,xcolor}
\usepackage{url,multirow,framed}
\usepackage[unicode=true,hidelinks,pdfusetitle,bookmarks,hypertexnames=false,debug]{hyperref}
\usepackage{listings}
\usepackage{type1cm}
\usepackage{cite}
\usepackage{pgfplots}
\usepackage{csvsimple}
\pgfplotsset{compat=1.7}
\bibliographystyle{junsrt}
\hypersetup{
   colorlinks=true,
   citecolor=black,
   linkcolor=black,
   urlcolor=blue
}
   \usetikzlibrary{intersections,calc,arrows.meta,backgrounds,shapes.geometric,shapes.misc,positioning,fit,graphs,arrows,plotmarks,fpu,datavisualization}
   \setlength{\columnsep}{5mm}
   \columnseprule=0.1mm
   \renewcommand{\indent}{1\zw}
   \setlength{\parindent}{1\zw}
   \ltjsetparameter{jacharrange={-2}} %日本語以外を欧文扱い
   \renewcommand{\thefootnote}{*\arabic{footnote}}
\renewcommand{\lstlistingname}{}
\renewcommand{\figurename}{}
\renewcommand{\tablename}{}
\AtBeginDocument{
           \renewcommand{\thelstlisting}{src \thechapter.\arabic{lstlisting}}
}
\lstset{
       %プログラム言語(複数の言語に対応,C,C++も可)
   language = {[x86masm]Assembler},
       %背景色と透過度
   %backgroundcolor={\color[gray]{.90}},
       %枠外に行った時の自動改行
   breaklines = true,
       %自動改行後のインデント量(デフォルトでは20[pt])
   breakindent = 10pt,
       %標準の書体
   basicstyle = \ttfamily\small,
       %コメントの書体
   commentstyle = {\ttfamily \color[cmyk]{1,0.4,1,0}},
       %関数名等の色の設定
   classoffset = 0,
       %キーワード(int, ifなど)の書体
   keywordstyle = {\bfseries \color[cmyk]{0,1,0,0}},
       %表示する文字の書体
   stringstyle = {\ttfamily \color[rgb]{0,0,1}},
       %枠 tは上に線を記載, Tは上に二重線を記載
       %他オプション:leftline,topline,bottomline,lines,single,shadowbox
   frame = lines,
       %frameまでの間隔(行番号とプログラムの間)
   framesep = 5pt,
       %行番号の位置
   numbers = left,
       %行番号の間隔
   stepnumber = 1,
       %行番号の書体
   numberstyle = \small,
       %タブの大きさ
   tabsize = 4,
       %キャプションの場所(tbならば上下両方に記載)
   captionpos = t,
   moredelim=**[is][\color{black}]{~}{~},
}
\lstdefinelanguage{Bash}{
   morekeywords=[1]{
       nasm,ld,uname,time,java
   },
   sensitive=true
}
\makeatletter
   \renewcommand{\thefigure}{%
   Fig\ \thechapter.\arabic{figure}}
   \@addtoreset{figure}{chapter}
   \renewcommand{\thetable}{%
   Tbl\ \thechapter.\arabic{table}}
   \@addtoreset{table}{chapter}

   \@addtoreset{lstlisting}{chapter}
   \newcommand{\figcaption}[1]{\def\@captype{figure}\caption{#1}}
   \newcommand{\tblcaption}[1]{\def\@captype{table}\caption{#1}}
\makeatother
\title{\vspace{-2cm}{\normalsize 情報学群実験第2 レポート}\\\vspace{0.5em}アセンブリ言語による整列アルゴリズムに関する実験}
\author{1250373 溝口洸熙\thanks{高知工科大学 情報学群 2年 清水研究室}}
\date{\today}
\renewcommand{\thechapter}{\arabic{chapter}}
\renewcommand{\thesection}{\arabic{section}}
\newcommand{\testsort}{{\ttfamily test\_sort.s}}
\newcommand{\sort}{{\ttfamily sort}}
\newcommand{\print}{{\ttfamily print\_eax}}
\begin{document}
\maketitle
\clearpage
\pagenumbering{roman}\pagestyle{plain}
\renewcommand{\lstlistlistingname}{ソースコード}
\tableofcontents
\listoffigures
\listoftables
\lstlistoflistings
\clearpage

\setcounter{page}{0}
\pagenumbering{arabic}
\chapter{即値代入の制限における規則性}
\section{実験の目的}
\aasm の{\ttfamily mov}命令\footnote{レジスタに即値,またはレジスタの値を代入する命令.}は,\iasm で扱った{\ttfamily mov}命令と対応する.しかし,\aasm の{\ttfamily mov}命令において,レジスタに代入できる即値には制限がある.本実験の目的は,その制限について実験に基づき示すことである.\par
\section{実験の方法}
コンピュータ(\ref{src:usingpc})を利用して,{\ttfamily mov.s}(\ref{src:mov})ファイルをアセンブルする.
\begin{lstlisting}[caption={利用コンピュータ及び実行環境},label={src:usingpc},language={Bash},numbers={none},breakindent={0pt}]
$ uname -a
Linux KUT20VLIN-322 5.4.0-70-generic #78~18.04.1-Ubuntu SMP Sat Mar 20 14:10:07 UTC 2021 x86_64 x86_64 x86_64 GNU/Linux
$ arm-none-eabi-as --version
GNU assembler (2,27-9ubuntu1+9) 2.27
$ bash --version
GNU bash, バージョン 4.4.20(1)-release (x86_64-pc-linux-gnu)
\end{lstlisting}
\begin{tabular}[c]{cc}
    \begin{minipage}[t]{0.45\textwidth}
        \centering
        \begin{lstlisting}[caption={アセンブル},label={src:assmeble},language={Bash},frame={left}]
$ arm-none-eabi-as mov.s -o mov.o
$ arm-none-eabi-ld mov.o -o mov
$ ./mov ; echo $?
3
    \end{lstlisting}
        \begin{flushleft}
            \ref{src:assmeble}: 4行目は,\ref{src:mov}: 5行目のテスト値を出力する.
        \end{flushleft}
    \end{minipage} &
    \begin{minipage}[t]{0.45\textwidth}
        \centering
        \begin{lstlisting}[caption={{\ttfamily mov.s}},label={src:mov},frame={left}]
    .section    .text
    .global     _start
_start:
    mov r7, #1
    mov r0, #3 @ test number
    swi #0
\end{lstlisting}
    \end{minipage}
    \vspace{0.5em}                                                                          \\
    \begin{minipage}[c]{0.45\textwidth}
        アセンブル(\ref{src:assmeble})の際に\ref{src:error}の出力が得られた場合は,\ref{src:mov}: 5行目に問題があり即値の代入に失敗している.
    \end{minipage}
    \hspace{1em}                                                                          &
    \begin{minipage}[c]{0.45\textwidth}
        \begin{lstlisting}[numbers={none},caption={Error出力},label={src:error},frame={single}]
mov.s: Assembler messages:
mov.s:5: Error: invalid constant (4d1) after fixup
    \end{lstlisting}
    \end{minipage}
\end{tabular}
様々なテスト値を簡単に試すためにスクリプトファイルによる自動実行プログラム(\ref{src:test})を作成し,実行した\footnote{実行するに際して,空ファイル{\ttfamily test.s}を作成し,{\ttfamily mov.s}の一部を変更する必要がある.}(\ref{src:testexec}).
テストする即値は,\(1\leq N\leq 15512\)である.実行結果には,アセンブルできた即値とその即値の前にアセンブルできた即値との差(前項との差)が出力される.
\begin{center}
    \begin{minipage}[t]{0.34\textwidth}
        \centering
        \begin{lstlisting}[caption={{\ttfamily mov.s}変更後},label={src:mov2},frame={left}]
    .equ    N,  %d
    .section    .text
    .global     _start
_start:
    mov r7, #1
    mov r0, #N @ test number
    swi #0
    \end{lstlisting}
        \begin{lstlisting}[language={Bash},numbers={none},caption={実行},label={src:testexec},frame={leftline}]
$ bash test.sh          
    \end{lstlisting}
    \end{minipage}
    \hspace{2em}
    \begin{minipage}[t]{0.58\textwidth}
        \centering
        \begin{lstlisting}[language={Bash},caption={{\ttfamily test.sh}},label={src:test},frame={left}]
#!/bin/bash
W=0
for i in 'seq 0 266240'
do
sed -e "s/%d/$i/g" mov.s > test.s
arm-none-eabi-as test.s -o test.o > /dev/null 2>&1
if [ $? -eq 0]; then
    echo "$i, OK, $(( $i - $W ))"
    V=$i
    W-$i
fi
done
    \end{lstlisting}
    \end{minipage}
\end{center}
\section{実験結果}
実験結果を\ref{tbl:実験結果シェル},\ref{tbl:実験結果手入力}に示す.\(N(n)\)は32ビットの\(n\)をビット反転させた数値である.
\begin{table}[H]
    \centering
    \begin{minipage}[t]{0.45\textwidth}
        \centering
        \caption{実験結果(シェルスクリプト)}
        \label{tbl:実験結果シェル}
        \scalebox{0.8}{
            \begin{tabular}{lccc}
                \multicolumn{2}{c}{入力数値} & アセンブルの可否     & \multicolumn{1}{c}{前項との差}          \\
                \hline
                0                        &              & OK                        & -      \\
                1                        &              & OK                        & 1      \\
                2                        &              & OK                        & 1      \\
                \vdots                   &              & \vdots                    & \vdots \\
                256                      & \((2^8)\)    & OK                        & 1      \\
                260                      &              & OK                        & 4      \\
                \vdots                   &              & \vdots                    & \vdots \\
                1024                     & \((2^{10})\) & OK                        & 4      \\
                1040                     &              & OK                        & 16     \\
                \vdots                   &              & \vdots                    & \vdots \\
                4096                     & \((2^{12})\) & OK                        & 16     \\
                4160                     &              & OK                        & 64     \\
                \vdots                   &              & \vdots                    & \vdots \\
                16384                    & \((2^{14})\) & OK                        & 64     \\
                16640                    &              & OK                        & 256    \\
                \vdots                   &              & \vdots                    & \vdots \\
                65536                    & \((2^{16})\) & OK                        & 256    \\
                66560                    &              & OK                        & 1024   \\
                \vdots                   &              & \vdots                    & \vdots \\
                262144                   & \((2^{18})\) & OK                        & 1024   \\
                266240                   &              & OK                        & 4096   \\
                \hline
            \end{tabular}
        }
    \end{minipage}
    \begin{minipage}[t]{0.45\textwidth}
        \centering
        \caption{実験結果(手入力)}
        \label{tbl:実験結果手入力}
        \scalebox{0.8}{
            \begin{tabular}{lc}
                \multicolumn{1}{c}{入力数値} & アセンブルの可否 \\
                \hline
                {\ttfamily 0x7fffff}     & MISS     \\
                {\ttfamily 0xfffffe}     & MISS     \\
                {\ttfamily 0xffffff}     & OK       \\
                \(N(2^{8})\)             & OK       \\
                \(N(2^{10})\)            & OK       \\
                \(N(2^{14})\)            & OK       \\
                \(N(0)\)                 & OK       \\
                \hline
            \end{tabular}
        }
        \begin{flushleft}
            \ref{tbl:実験結果シェル}を入力数値\(n\in\{0\}\cup\mathbb{Z_+}\)に対して,アセンブルの可能条件を一般化する.
            \begin{tcolorbox}[
                    enhanced,
                    title={\bfseries \hypertarget{kisoku1}{規則 1}},
                    attach boxed title to top left={xshift=3mm,yshift*=-\tcboxedtitleheight/2},
                    sharp corners
                ]
                \begin{align*}
                    2^p< n\leq 2^{p+2} &    & (p=2k+6)
                    \intertext{となる\(k(1\leq k\leq 5)\)に対して,}
                    n\bmod 4^k         & =0
                \end{align*}
                であることが,\(n\)をアセンブルできる条件である.
            \end{tcolorbox}
        \end{flushleft}
    \end{minipage}
\end{table}
\section{考察}
\newcommand{\kisokua}{\hyperlink{kisoku1}{\bfseries 規則1}}
\paragraph{結果からの考察}\ref{tbl:実験結果シェル}より,{\ttfamily mov}命令による18ビットまでの即値の代入に関しては\kisokua に従っていることが判る.
さらに,入力数値のビット列が8ビットに収まり,その8ビットに収まったビット列の先頭と,レジスタの1ビット目との距離が2の倍数である必要がある(\ref{fig:exec}).\par
また\ref{tbl:実験結果手入力}より,入力数値\(i\)が{\ttfamily 0xffffff}より大きい場合,\(N(i)\)が\kisokua に準拠するならばアセンブルが可能であることが予想される(\ref{fig:exec2}).\par
つまり,入力数値\(n\in\{0\}\cup\mathbb{Z_+}\)に対して,アセンブルの可能条件は以下のようになることが予想される.
\newcounter{reg}
\setcounter{reg}{-1}
\newcommand{\reg}[1][]{\refstepcounter{reg}\arabic{reg}}
\begin{figure}[h]
    \caption{アセンブル可能な即値の例1}
    \label{fig:exec}
    \begin{center}
        \begin{tikzpicture}
            \fill[fill=magenta!10,opacity=.5](4,0)rectangle(8,1);
            \draw[thick](4,0)rectangle(8,1);
            \draw (0,0) rectangle (16,1);
            \foreach \u in {0.5,1.0, ..., 16.0}
            \draw (\u,0)--(\u,0.2);
            \foreach \u in {0.5,1.0, ..., 16.0}
            \draw (\u,1)--(\u,0.8);
            \foreach \u in {4.25,5.25,6.75,7.75}
            \node at (\u,0.5){{\ttfamily 1}};
            \foreach \u in {4.75,5.75,6.25,7.25}
            \node at (\u,0.5){{\ttfamily 0}};
            \foreach \u in {0.25,0.75,...,3.75}
            \node at (\u,0.5){{\ttfamily 0}};
            \foreach \u in {15.75,15.25,...,0.25}
            \node[above] at (\u,1){\scriptsize\ttfamily \reg};
            \foreach \u in {15.75,15.25,...,8.25}
            \node at (\u,0.5){{\ttfamily 0}};
            \draw[latex-latex,thick](8,-0.5)--(16,-0.5)node[midway,fill=white]{\scriptsize 2の倍数ビット};
            \draw[dashed](8,0)--(8,-1);
            \draw[dashed](16,0)--(16,-1);
            \node[below] at ($(4,0)!0.5!(8,0)$){\scriptsize 8ビットに収まっている};
        \end{tikzpicture}
    \end{center}
    \caption{アセンブル可能な即値の例2}
    \label{fig:exec2}
    \begin{center}
        \begin{tikzpicture}
            \fill[fill=magenta!10,opacity=.5](4,0)rectangle(8,1);
            \draw[thick](4,0)rectangle(8,1);
            \draw (0,0) rectangle (16,1);
            \foreach \u in {0.5,1.0, ..., 16.0}
            \draw (\u,0)--(\u,0.2);
            \foreach \u in {0.5,1.0, ..., 16.0}
            \draw (\u,1)--(\u,0.8);
            \foreach \u in {4.25,5.25,6.75,7.75}
            \node at (\u,0.5){{\ttfamily 0}};
            \foreach \u in {4.75,5.75,6.25,7.25}
            \node at (\u,0.5){{\ttfamily 1}};
            \foreach \u in {0.25,0.75,...,3.75}
            \node at (\u,0.5){{\ttfamily 1}};
            \foreach \u in {15.75,15.25,...,0.25}
            \node[above] at (\u,1){\scriptsize\ttfamily \reg};
            \foreach \u in {15.75,15.25,...,8.25}
            \node at (\u,0.5){{\ttfamily 1}};
            \draw[latex-latex,thick](8,-0.5)--(16,-0.5)node[midway,fill=white]{\scriptsize 2の倍数ビット};
            \draw[dashed](8,0)--(8,-1);
            \draw[dashed](16,0)--(16,-1);
            \node[below] at ($(4,0)!0.5!(8,0)$){\scriptsize 8ビットに収まっている};
        \end{tikzpicture}
    \end{center}
    \caption{アセンブルが不可能な即値の例1}
    \setcounter{reg}{-1}
    \begin{center}
        \begin{tikzpicture}
            \fill[fill=magenta!10,opacity=.5](4,0)rectangle(8.5,1);
            \draw[thick](4,0)rectangle(8.5,1);
            \draw (0,0) rectangle (16,1);
            \foreach \u in {0.5,1.0, ..., 16.0}
            \draw (\u,0)--(\u,0.2);
            \foreach \u in {0.5,1.0, ..., 16.0}
            \draw (\u,1)--(\u,0.8);
            \foreach \u in {4.25,5.25,6.75,7.75,8.25}
            \node at (\u,0.5){{\ttfamily 1}};
            \foreach \u in {4.75,5.75,6.25,7.25}
            \node at (\u,0.5){{\ttfamily 0}};
            \foreach \u in {15.75,15.25,...,0.25}
            \node[above] at (\u,1){\scriptsize\ttfamily \reg};
            \foreach \u in {15.75,15.25,...,8.75}
            \node at (\u,0.5){{\ttfamily 0}};
            \foreach \u in {0.25,0.75,...,3.75}
            \node at (\u,0.5){{\ttfamily 0}};
            \draw[latex-latex,thick](8.5,-0.5)--(16,-0.5)node[midway,fill=white]{\scriptsize 2の倍数ビットではない};
            \draw[dashed](8.5,0)--(8.5,-1);
            \draw[dashed](16,0)--(16,-1);
            \node[below] at ($(4,0)!0.5!(9,0)$){\scriptsize 8ビットより大きい};
        \end{tikzpicture}
    \end{center}
    \caption{アセンブルが不可能な即値の例2}
    \setcounter{reg}{-1}
    \begin{center}
        \begin{tikzpicture}
            \fill[fill=magenta!10,opacity=.5](4,0)rectangle(8.5,1);
            \draw[thick](4,0)rectangle(8.5,1);
            \draw (0,0) rectangle (16,1);
            \foreach \u in {0.5,1.0, ..., 16.0}
            \draw (\u,0)--(\u,0.2);
            \foreach \u in {0.5,1.0, ..., 16.0}
            \draw (\u,1)--(\u,0.8);
            \foreach \u in {4.25,5.25,6.75,7.75,8.25}
            \node at (\u,0.5){{\ttfamily 0}};
            \foreach \u in {4.75,5.75,6.25,7.25}
            \node at (\u,0.5){{\ttfamily 1}};
            \foreach \u in {15.75,15.25,...,0.25}
            \node[above] at (\u,1){\scriptsize\ttfamily \reg};
            \foreach \u in {15.75,15.25,...,8.75}
            \node at (\u,0.5){{\ttfamily 1}};
            \foreach \u in {0.25,0.75,...,3.75}
            \node at (\u,0.5){{\ttfamily 1}};
            \draw[latex-latex,thick](8.5,-0.5)--(16,-0.5)node[midway,fill=white]{\scriptsize 2の倍数ビットではない};
            \draw[dashed](8.5,0)--(8.5,-1);
            \draw[dashed](16,0)--(16,-1);
            \node[below] at ($(4,0)!0.5!(9,0)$){\scriptsize 8ビットより大きい};
        \end{tikzpicture}
    \end{center}
\end{figure}
\newpage
\paragraph{\aasm の{\ttfamily mov}命令の詳細}
\begin{wrapfigure}{r}[0pt]{0.35\textwidth}
    \caption{即値の利用}
    \label{fig:即値の利用}
    \setcounter{reg}{-1}
    \begin{tikzpicture}
        \fill[cyan!10](0,0) rectangle (2,1);
        \fill[yellow!10](2,0) rectangle (6,1);
        \draw (0,0) rectangle (6,1);
        \foreach \u in {0.5,1.0,...,5.5}
            {
                \draw (\u,0)--(\u,0.2);
                \draw (\u,1)--(\u,0.8);
            }
        \foreach \u in {5.75,5.25,...,0.25}
        \node[above] at (\u,1){\scriptsize\ttfamily \reg};
        \node at ($(0,0)!0.5!(2,1)$){\small 回転数\(r\)};
        \node at ($(2,0)!0.5!(6,1)$){\small 即値\(I\)};
    \end{tikzpicture}
    \caption{結果}
    \label{fig:結果}
    \setcounter{reg}{-1}
    \begin{tikzpicture}
        \fill[cyan!10](0,0) rectangle (2,1);
        \fill[yellow!10](2,0) rectangle (6,1);
        \draw (0,0) rectangle (6,1);
        \foreach \u in {0.5,1.0,...,5.5}
            {
                \draw (\u,0)--(\u,0.2);
                \draw (\u,1)--(\u,0.8);
            }
        \foreach \u in {5.75,5.25,...,0.25}
        \node[above] at (\u,1){\scriptsize\ttfamily \reg};
        \foreach \u \v in {2.25/1,2.75/0,3.25/1,3.75/0,4.25/0,4.75/1,5.25/0,5.75/1,0.25/1,0.75/0,1.25/0,1.75/0}
        \node at (\u,0.5) {\ttfamily \v};
    \end{tikzpicture}
\end{wrapfigure}
資料\cite{armasm}によると,ARMデータの処理命令ビットレイアウトで,即値に当てられているのは下位12ビットのみである.\par
さらに,12ビットの即値を12ビットの番号として処理するのではなく,4ビットの回転値と8ビットの値として処理する(\ref{fig:即値の利用}).
この回転値\(r\)は即値\(I\)を右へ\(2r\)ローテートするために用いる.\par
仮に\(i=\){\ttfamily\ 0x00D30000}が入力されたとする.\(i\)のビット列と右ローテートを\ref{fig:ex}に示す.\(i\)はそのままだと即値として利用できないため,即値として利用できる下位8ビット\(I\)にローテートさせる.(シフトではない.)
このとき16ビットシフトさせたのでこの値を2で割った8を回転数\(r\)に格納し,下位12ビットのみで23ビット必要である数値\(i=\){\ttfamily\ 0x00D30000}を\ref{fig:結果}のように表すことができる.
\begin{figure}[H]
    \centering
    \caption{\(i\)のビット列,右ローテート}
    \label{fig:ex}
    \setcounter{reg}{-1}
    \begin{tikzpicture}[remember picture]
        \fill[fill=magenta!10,opacity=.5](4,0)rectangle(8,1);
        \draw[thick](4,0)rectangle(8,1);
        \draw (0,0) rectangle (16,1);
        \foreach \u in {0.5,1.0, ..., 16.0}
        \draw (\u,0)--(\u,0.2);
        \foreach \u in {0.5,1.0, ..., 16.0}
        \draw (\u,1)--(\u,0.8);
        \foreach \u in {4.25,5.25,6.75,7.75}
        \node at (\u,0.5){{\ttfamily 1}};
        \foreach \u in {4.75,5.75,6.25,7.25}
        \node at (\u,0.5){{\ttfamily 0}};
        \foreach \u in {0.25,0.75,...,3.75}
        \node at (\u,0.5){{\ttfamily 0}};
        \foreach \u in {15.75,15.25,...,0.25}
        \node[above] at (\u,1){\scriptsize\ttfamily \reg};
        \foreach \u in {15.75,15.25,...,8.25}
        \node at (\u,0.5){{\ttfamily 0}};
        \coordinate (S) at (8,0);
        \coordinate (G) at (4,0);
        \fill[fill=magenta!10,opacity=.5](12,-1)rectangle(16,-2);
        \draw[thick](12,-1)rectangle(16,-2);
        \draw (0,-1) rectangle (16,-2);
        \foreach \u in {0.5,1.0, ..., 16.0}
        \draw (\u,-1)--(\u,-1.2);
        \foreach \u in {0.5,1.0, ..., 16.0}
        \draw (\u,-2)--(\u,-1.8);
        \foreach \u in {15.75,14.75,13.25,12.25}
        \node at (\u,-1.5){{\ttfamily 1}};
        \foreach \u in {15.25,14.25,13.75,12.75}
        \node at (\u,-1.5){{\ttfamily 0}};
        \foreach \u in {0.25,0.75,...,11.75}
        \node at (\u,-1.5){{\ttfamily 0}};
        \foreach \u in {15.75,15.25,...,0.25}
        \node[below] at (\u,-2){\scriptsize\ttfamily \reg};
        \coordinate (G2) at (12,-1);
        \coordinate (S2) at (16,-1);
        \draw(G)--(G2);
        \draw(S)--(S2);
    \end{tikzpicture}
\end{figure}
\paragraph{このような設計になっている理由} ARMデータ処理命令のビットレイアウト\cite{armasm}によると,1つのレジスタ内に{\ttfamily Cond}や,加減算,移動,比較などを正確に定義するための値も格納されている.32ビット全てを即値に当てることができず,より少ないビット数でより多くの数値を表現するためにこのような設計になったのだろう.
この設計においての不都合は,幾つかの即値を{\ttfamily mov}命令において代入できないことだが,\(2^n(n\in\mathbb{N},0\leq n\leq 31)\)や,32ビット・ワード内の任意の4バイトの位置にあるバイト値の代入を保証しており,最も一般的なケースをカバーしているので使用上の問題は少ないと言える.\cite[p.52]{armprocesser}
\chapter{アセンブリ言語・高級言語の実装の違い}\label{cha2}
\section{実験の目的}
先に述べたように,アセンブリ言語は機械語を1対1に記した記法である.それに対して高級言語(高水準言語)は,人間の言語・概念に近づけて設計されてたプログラム言語である.\cite{高水準言語}\par
本実験の目的は,アセンブリ言語や機械語などの低級言語は{\ttfamily Java}などの高級言語と比べて,実装に関して大きく異なる点があるかどうかを明らかにすること,すなわち,コードの複雑さやコード量がどうなるかを明らかにすることである.
\section{実験の方法}
実装に関して大きく異なる点を明らかにするため,低級言語と高級言語の一般的な実装手順を書き出し比較をする.低級言語はアセンブリ言語,高級言語は{\ttfamily Java}を利用する.\par
さらに,コードの複雑さやコード量がどうなるかを明らかにするため,実際に2つの言語で書いた同一のアルゴリズムに対して,行数や条件分岐,ループの回数を比べる.\par
{\ttfamily Java}におけるループ回数は{\ttfamily for}文の個数,比較回数は{\ttfamily if}文の個数とし,アセンブリ言語における比較回数は{\ttfamily cmp}の個数,ループ回数は一定条件下で上の行のラベルにジャンプする回数とする.評価値は
\[評価値=行数+比較回数+ループ回数\]と定義し,評価値とコードの複雑さは比例するものとする.
一般的にコードの行数が短いことと,複雑であるか否かは必要十分ではない.ただ,アセンブリ言語は1行につき1命令で記述する必要があるため,行数の比較で十分であると判断した.
\section{実験の結果}
\ref{src:sort.s},\ref{src:testsort.s}は,アセンブリ言語で記述したプログラム,\ref{src:sort.java}は{\ttfamily Java}で記述したプログラムである.いずれも,入力データを受け取り選択ソートアルゴリズムで整列してその整列結果を1行ずつ出力するプログラムであり,入力と出力は一致している.それぞれの行数と比較回数を\ref{tbl:行数の比較}に示す.\par
\begin{table}[H]
   \centering
   \caption{行数とループ・比較回数}
   \label{tbl:行数の比較}
   \begin{tabular}{p{4cm}p{3cm}wc{2cm}wc{2cm}wc{2cm}wc{2cm}}
       \multicolumn{1}{c}{記述言語} & \multicolumn{1}{c}{ファイル名} & \multicolumn{1}{c}{行数} & \multicolumn{1}{c}{比較回数} & \multicolumn{1}{c}{ループ回数} & 評価値                 \\
       \hline
       \multirow{2}{*}{アセンブリ言語} & {\ttfamily sort.s}        & 44                     & 3                        & 2                         & \multirow{2}{*}{77} \\
                                & {\testsort}               & 26                     & 1                        & 1                                               \\
       \hline
       {\ttfamily Java}         & {\ttfamily sort.java}     & 19                     & 1                        & 2                         & 22                  \\
       \hline
   \end{tabular}
\end{table}
\section{考察}
実験の結果より,両言語の評価値を比べるとアセンブリ言語の評価値の方が{\ttfamily Java}に比べて3.5倍であることが確認できる.1番目立った違いは行数であろう.アセンブリ言語で記述したものに比べて{\ttfamily Java}で記述したアルゴリズムは約\(1/3\)とより簡潔に記述できることが分かる.\par
その原因として,{\ttfamily Java}のループに使われる{\ttfamily for}文は,比較とジャンプ,ループ変数の定義と処理を1行で行うことが可能であることに対して,アセンブリ言語では比較・ジャンプ・ループ変数の処理の命令を1つずつ記述する必要のあることが挙げられる.\par
今回の実験で,低級言語であるアセンブリ言語の方が,高級言語である{\ttfamily Java}よりも複雑でゴードの量も多くなることが分かった.ただし,この実験では独自の指標でコードの複雑さを測っているため,一般的な複雑の指標である,サイクロマティック複雑度(循環的複雑度)で計測できていない.

\chapter{低級言語での理論的計算量に関する実験}\label{chap:時間計測}
\section{実験の目的}
今回の実験の目的は,アセンブリ言語・機械語で直接記述した場合も実行時間は論理的計算量(アルゴリズムによって\(O(n^2)\)や\(O(n\log n)\)など)に従うことを示すことである.選択ソートのアルゴリズムの計算量は\(O(n^2)\)である\cite[p.50,51]{アルゴリズムとデータ構造}ので,アセンブリ言語で記述した選択ソートアルゴリズムもそれに従うか検証するれば,目的を満たすことになる.
\section{実験方法}\label{sec:時間計測}
より正確に選択ソートの実行時間を計測するため,\testsort ファイルを\print を利用する箇所を削除し,\ref{src:データの個数指定}に書き換える.\par
\begin{wrapfigure}{r}[1pt]{0.3\textwidth}
    \vspace{-3em}
    \begin{lstlisting}[language={Bash},caption={時間計測実行コマンド},label={src:command3},frame={single},numbers={none}]
$ time ./a.out
-- 実行結果(略)--
real    0m0.002s
user    0m0.001s
sys     0m0.000s
\end{lstlisting}
    \vspace{-3em}
\end{wrapfigure}
実験環境は,\ref{usingPC}に示した通り.アセンブリ言語での実行時間の検証は,Linux標準の{\ttfamily time}コマンドを用いて計測する.\ref{command1}のコマンドを実行し,実行ファイル{\ttfamily a.out}を生成した後,\ref{src:command3}を実行し実行時間を計測する.
各実行時間の中でも{\ttfamily real}が引数コマンドを実行するのにかかった時間である.実験回数は1つのテストにつき3回行い実行時間を平均する.
さらに,選択ソートは最良時間計算量と最悪時間計算量が等しく\cite[p.50]{アルゴリズムとデータ構造},整列対象のデータ列は計測時間に依存しない故,今回はデータ列を全て{\ttfamily 0}に定める.
実験結果をExcelを使って多項式近似(2次)を求め,その関数曲線と実験結果を比較する.テストデータ数は\eqref{equ:テストデータ}の集合\(T\)である.\\
\begin{minipage}[c]{0.43\textwidth}
    \centering
    \hspace*{2em}
    \begin{lstlisting}[caption={\testsort 書き換え後}, label={src:データの個数指定},frame={single},numbers={none}]
略
    extern  sort
_start:
mov ebx,    data
mov ecx,    ndata
call sort
mov eax,    1
mov ebx,    0
int 0x80
    section .data
data: times データ個数 dd 0
ndata: equ ($ - data) / 4
    \end{lstlisting}
\end{minipage}
\begin{minipage}[c]{0.49\textwidth}
    \centering
    \begin{align}
        T_1 & =\{10^n\mid n\in\mathbb{N}, n\leq 5\} \notag \\
        T_2 & =\{n\times 10^4\mid 2\leq n\leq 9\}  \notag  \\
        T   & =T_1\cup T_2\label{equ:テストデータ}
    \end{align}
    \hspace{1em}
\end{minipage}
\section{実験結果}
実験結果\ref{fig:データの個数と実行時間の曲線グラフ}に示す.実行時間実験結果詳細は,\ref{tblr:実行時間計測実験結果A}に掲載している.
\begin{figure}[htb]
    \centering
    \caption{アセンブリ言語のデータの個数と実行時間の曲線グラフ}
    \label{fig:データの個数と実行時間の曲線グラフ}
    \begin{tikzpicture}
        \begin{axis}[
                xlabel={データ個数(個)\(n\)},
                ylabel={実行時間(秒)\(t\)},
                enlarge x limits=false,
                width=0.9\textwidth,
                height=0.3\textheight,
                ytick distance=0.5,
                ymin=0,
                legend = inner,
                legend pos=north west,
            ]
            \addplot[black,thick,mark=*] table[x=n, y=av, col sep=comma]{exdataA_all.csv};
            \addlegendentry{実験結果}
            \addplot[dashed,domain=0:100000,thin,mark=none,samples={300}] plot(\x, {6 * pow(10,-10) * (\x)^2 + 9 * pow(10,-7) * (\x) - 0.0004});
            \addlegendentry{\(t=6\cdot 10^{-10}n^2+9\cdot 10^{-7}n-0.0004\)}
        \end{axis}
    \end{tikzpicture}
\end{figure}
\begin{align}
    \intertext{実行時間を\(t\),データの個数を\(n\)とすると,近似多項式は\eqref{equ:近似多項式}.}
    t= 6\cdot 10^{-10}n^2+9\cdot 10^{-7}n-0.0004\label{equ:近似多項式}
\end{align}
\section{考察}
オーダ記法は,アルゴリズムの時間計算量の入力サイズ\(n\)を用いた関数\(f(n)\)に対して,その関数の主要項の係数を削除した\(O(f(n))\)である.\cite[p.7]{アルゴリズムとデータ構造}\par
従って,\eqref{equ:近似多項式}を元にオーダ記法の定義より,このアルゴリズムの計算量は\(O(n^2)\)であることが分かる.
\chapter{高級言語と低級言語の計算時間に関する実験}\label{chap4}
\section{実験の目的}
アセンブリ言語と高級言語での実装を比べて計算時間に差があるかどうかを明らかにする.
\section{実験の方法}
データ数は,\ref{src:sort.java}の3行目を
\begin{lstlisting}[caption={}, label={}, language={Java}, frame={none},numbers={none}]
int[] data = new int[データ数];
\end{lstlisting}
とすることでデータ数に応じた実験ができる.\par
実験環境は,\ref{usingPC}に示した通りである.{\ttfamily Java}のコンパイル・実行は\ref{src:javac}に示す.
\begin{lstlisting}[caption={{\ttfamily Java}コンパイル・実行時間の計測}, label={src:javac}, language={Bash},frame={single},numbers={none}]
$ javac sort.java
$ time java sort
\end{lstlisting}
\section{実験結果}
実験結果を\ref{fig:比較}に示す.{\ttfamily Java}の実行時間実験結果詳細は,\ref{tblr:実行時間計測実験結果J}に掲載している.
\begin{figure}[htb]
   \centering
   \caption{各言語の実行時間比較}
   \label{fig:比較}
   \begin{tikzpicture}
       \begin{axis}[
               ylabel={実行時間(秒)},
               xlabel={データ個数(個)},
               xmin = 0, ymin=0,
               width=0.8\textwidth,
               height=0.27\textheight,
               legend=inner,
               legend pos=north west,
               ytick distance=0.5,
               xmax=100000,
               yticklabel style={font=\small},
           ]
           \addplot[black,thick,mark=*] table[x=n,y=av,col sep=comma]{exdataA_all.csv};
           \addlegendentry{アセンブリ言語}
           \addplot[black,thick,dashed,mark=square*,mark options={solid}]table[x=n,y=av, col sep=comma]{exdataJ_all.csv};
           \addlegendentry{{\ttfamily Java}}
       \end{axis}
   \end{tikzpicture}
\end{figure}
\section{考察}
実験結果より,アセンブリ言語で記述した選択ソートの方が{\ttfamily Java}で記述した選択ソートよりも実行時間は短いことがわかった.
ただし,いずれも\ref{src:sort.java},\ref{src:sort.s}のアルゴリズムでの実行時間であるので,一般的にアセンブリ言語の方が{\ttfamily Java}よりも実行時間が短いとの結論にはならない.\par
{\ttfamily Java}はクラスファイルからの実行はインタプリタ方式をとっているため,機械語に翻訳された実行ファイルよりも実行に時間がかかると考えらえる.

\chapter{アセンブリ言語による記述の利点}
\section{目的}
\ref{chap1}章から\ref{chap4}章で得られた事実や考察に基づいて,アセンブリ言語で整列アルゴリズムを直接記述することに利点があるかどうかを明らかにする.
\section{考察}
\ref{chap4}章の実験結果より,高級言語である{\ttfamily Java}よりもアセンブリ言語で記述した選択アルゴリズムの方が,実行時間は短いことが分かっている.
ただ,先にも述べたように{\ttfamily Java}がコンパイラ・インタプリタ方式採用していることが大きな原因と考えられ,仮にコンパイラ方式を採用している言語で実験したとしても,一般に高級言語よりも低級言語の方が実行時間は短いとは断定できない.今回の実験に限ってはアセンブリ言語での記述によって整列アルゴリズムがより高速に実行できたので,この点についてはアセンブリ言語で記述する利点であろう.\par
アセンブリ言語はCPU内のレジスタの値の操作,主記憶領域の値の操作を1つずつ記入する.\ref{cha2}の実験を踏まえると,コード量は一般の高級言語に比べるととても多くなり複雑になる.
これは,プログラマーの人的ミスを引き起こしやすい要因でもあり,コードが複雑化するアセンブリ言語の欠点とも言える.

\chapter*{謝辞}
\pagestyle{sj}
本実験課題は本学情報学群 1250372 三上 柊氏と共同で実施した.また本学情報学群 1250383 山本 昇永氏にはシェルスクリプトファイル(\ref{src:test})の作成にご協力いただいた.
これらの方々に深く感謝いたします.
\begin{flushright}
    溝口 洸熙
\end{flushright}
\bibliography{bib}
\renewcommand{\thesection}{\Alph{section}}
\chapter*{付録}
\addcontentsline{toc}{chapter}{付録}
\setcounter{section}{0}
\section{ソースコード}
\renewcommand{\thelstlisting}{src \thesection.\arabic{lstlisting}}
\renewcommand{\thetable}{Tbl \thesection.\arabic{table}}
\begin{lstlisting}[caption={{\ttfamily sort.s}},label={src:sort.s},frame={shadowbox}]
   section .text
   global  sort
 sort:
   push  esi
   push  edi
   push  edx
   push  ecx
   push  ebx
   push  eax
   dec   ecx
 loop0:
   cmp   ecx,  0 ; ecx = i
   jle   endp
   mov   edx,  [ebx]   ; max = data[0]
   mov   eax,  0       ; max_indent = 0
   mov   edi,  1 ; edi = j
   loop1:
     cmp   edi,  ecx   ; j > i?
     jg    loop0l
     mov   esi,  [ebx + edi*4] ; data[j]
     cmp   esi,  edx       ; data[j] >= max
     jge   then
     jmp   endif
     then:
       mov edx,  [ebx + edi*4] ; max = data[j]
       mov eax,  edi           ; max_index = j
     endif:
       inc edi
       jmp loop1
   loop0l:
     mov   esi,  [ebx + eax*4]   ; m = data[max_index]
     mov   edi,  [ebx + ecx*4]   ; edi = data[i]
     mov   [ebx + eax*4],  edi   ; data[max_index],  data[i]
     mov   [ebx + ecx*4],  esi   ; data[i] = m
     dec   ecx
     jmp   loop0
 endp:
   pop eax
   pop ebx
   pop ecx
   pop edx
   pop edi
   pop esi
   ret
\end{lstlisting}
\begin{lstlisting}[caption={{\ttfamily test\_sort.java}},label={src:testsort.s},frame={shadowbox}]
 section .text
 global  _start
 extern  sort, print_eax

_start:
 mov ebx,  data
 mov ecx,  ndata
 call  sort      ; ソート
 mov edi,  0      
loop:   ; 結果の出力
 cmp edi,  ndata
 je  endp
 mov eax,  [data + edi * 4]
 call print_eax
 inc edi
 jmp loop


endp:
 mov eax,  1
 mov ebx,  0
 int 0x80

 section .data
data:  dd 1, 3, 5, 7, 9, 2, 4, 6, 8, 0, 1, 2
ndata  equ ($ - data) / 4  
\end{lstlisting}
\begin{lstlisting}[language={Java},caption={{\ttfamily sort.java}},label={src:sort.java},frame={shadowbox}]
class sort{
   public static void main(String[] args){
       int[] data = {1, 3, 5, 7, 9, 2, 4, 6, 8, 0, 1, 2};
       buble(data);
       for(int d: data){
           System.out.println(d);// 出力
       }
   }
   private static void buble(int[] data) {
       int max_index = 0; int max = 0;
       for (int i = data.length - 1; i > 0; i--) {
           max = data[0]; max_index = 0;
           for (int j = 1; j <= i; j++) {
               if (data[j] >= max) { max = data[j]; max_index = j; }
           }
           int m = data[max_index]; data[max_index] = data[i]; data[i] = m;
       }
   }
}
\end{lstlisting}

\section{実験結果の詳細}
\begin{table}[h]
   \centering
   \begin{minipage}[t]{0.49\linewidth}
       \centering
       \caption{実行時間計測実験結果 (アセンブリ言語)}
       \label{tblr:実行時間計測実験結果A}
       \scalebox{0.7}{
           \begin{tabular}{lllll}
               \hline
               \multicolumn{1}{c}{\multirow{2}{*}{データ個数}} & \multicolumn{3}{|c|}{実行時間(秒)} & \multicolumn{1}{c}{\multirow{2}{*}{平均}}                                   \\
               \cline{2-4}
                                                          & \multicolumn{1}{|c}{1回目}      & \multicolumn{1}{c}{2回目}                 & \multicolumn{1}{c|}{3回目} &      \\
               \hline
               \csvreader[late after line=\\]{exdataA_all.csv}
               {n=\nn,t1=\ta,t2=\tb,t3=\tc,av=\av}
               {\nn                                       & \ta                           & \tb                                     & \tc                      & \av}
               \hline
           \end{tabular}
       }
   \end{minipage}
   \begin{minipage}[t]{0.49\linewidth}
       \centering
       \caption{実行時間計測実験結果 ({\ttfamily Java})}
       \label{tblr:実行時間計測実験結果J}
       \scalebox{0.7}{
           \begin{tabular}{llllll}
               \hline
               \multicolumn{1}{c}{\multirow{2}{*}{データ個数}} & \multicolumn{3}{|c|}{実行時間(秒)} & \multicolumn{1}{c}{\multirow{2}{*}{平均}}                                     \\
               \cline{2-4}
                                                          & \multicolumn{1}{|c}{1回目}      & \multicolumn{1}{c}{2回目}                 & \multicolumn{1}{c|}{3回目} &        \\
               \hline
               \csvreader[late after line=\\]{exdataJ_all.csv}
               {n=\one,t1=\tow,t2=\three,t3=\four,av=\five}
               {\one                                      & \tow                          & \three                                  & \four                    & \five}
               \hline
           \end{tabular}
       }
   \end{minipage}
\end{table}

\end{document}
