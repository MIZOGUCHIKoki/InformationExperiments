\chapter{高級言語と低級言語の計算時間に関する実験}\label{chap4}
\section{実験の目的}
アセンブリ言語と高級言語での実装を比べて計算時間に差があるかどうかを明らかにする.
\section{実験の方法}
データ数は,\ref{src:sort.java}の3行目を
\begin{lstlisting}[caption={}, label={}, language={Java}, frame={none},numbers={none}]
int[] data = new int[データ数];
\end{lstlisting}
とすることでデータ数に応じた実験ができる.\par
実験環境は,\ref{usingPC}に示した通りである.{\ttfamily Java}のコンパイル・実行は\ref{src:javac}に示す.
\begin{lstlisting}[caption={{\ttfamily Java}コンパイル・実行時間の計測}, label={src:javac}, language={Bash},frame={single},numbers={none}]
$ javac sort.java
$ time java sort
\end{lstlisting}
\section{実験結果}
実験結果を\ref{fig:比較}に示す.{\ttfamily Java}の実行時間実験結果詳細は,\ref{tblr:実行時間計測実験結果J}に掲載している.
\begin{figure}[htb]
   \centering
   \caption{各言語の実行時間比較}
   \label{fig:比較}
   \begin{tikzpicture}
       \begin{axis}[
               ylabel={実行時間(秒)},
               xlabel={データ個数(個)},
               xmin = 0, ymin=0,
               width=0.8\textwidth,
               height=0.27\textheight,
               legend=inner,
               legend pos=north west,
               ytick distance=0.5,
               xmax=100000,
               yticklabel style={font=\small},
           ]
           \addplot[black,thick,mark=*] table[x=n,y=av,col sep=comma]{exdataA_all.csv};
           \addlegendentry{アセンブリ言語}
           \addplot[black,thick,dashed,mark=square*,mark options={solid}]table[x=n,y=av, col sep=comma]{exdataJ_all.csv};
           \addlegendentry{{\ttfamily Java}}
       \end{axis}
   \end{tikzpicture}
\end{figure}
\section{考察}
実験結果より,アセンブリ言語で記述した選択ソートの方が{\ttfamily Java}で記述した選択ソートよりも実行時間は短いことがわかった.
ただし,いずれも\ref{src:sort.java},\ref{src:sort.s}のアルゴリズムでの実行時間であるので,一般的にアセンブリ言語の方が{\ttfamily Java}よりも実行時間が短いとの結論にはならない.\par
{\ttfamily Java}はクラスファイルからの実行はインタプリタ方式をとっているため,機械語に翻訳された実行ファイルよりも実行に時間がかかると考えらえる.
