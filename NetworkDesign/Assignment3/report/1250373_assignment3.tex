% Compile : uplatex*2 -> dvipdfmx
\documentclass[12pt,dvipdfmx]{jlreq}
\usepackage{listings,xcolor,lastpage,framed}
\usepackage[left=10truemm,right=10truemm,top=20truemm,bottom=20truemm,headheight=22pt]{geometry}
\usepackage{fancyhdr}
\renewcommand{\texttt}[1]{{\ttfamily\bfseries\ #1\ }}
\fancyhf{}
\fancyhead[R]{\thepage\ / \pageref{LastPage}}
\fancyhead[C]{Assignment No.2}
\fancyhead[L]{1250373 溝口 洸熙}

\fancyfoot[R]{Compiled by Cloud LaTeX}
\fancyfoot[C]{Network-Design 2023}
\fancyfoot[L]{\fbox{Deadline: Feb. 1st, 2024 23:59}}
\pagestyle{fancy}
\lstset{ %ここで設定する内容は,全てのlstlisting環境で適用されます.
    %【凡例】コマンド名 = {設定値} % コマンドの意味 (デフォルト値 | 選択肢)
    language = {C}, %利用言語
    % --- コード書式設定 ---
    basicstyle = {\ttfamily\small}, % 基本書式
    commentstyle = {\color{green!50!black}}, % コメント書式
    keywordstyle = [1]{\color{red!50!black}\bfseries}, % キーワード書式:区分[1]
    keywordstyle = [2]{\color{red}\bfseries}, % キーワード書式:区分[2]
    keywordstyle = [3]{\color{orange}\bfseries}, % キーワード書式:区分[3]
    stringstyle = {\color{blue}}, % 文字列書式
    showspaces = {false}, % 空白表示 (false | true, false)
    showtabs = {false}, % Tabを表示 (false | true, false)
    tab = {}, % Tab記号( | $\mapsto$, $\to$など)
    % --- その他の設定 ---
    frame = {tl}, % フレーム設定 (none | leftline, topline, bottomline, lines, single, shadowbox, {T,t, B,b, L,l, R,r})
    framesep = {.1cm}, % フレームまでの間隔 ( 0cm | Npt, Ncmなど)
    breaklines = {true}, % 1行が長いとき,改行するか否かの設定 (false | true, false)
    breakindent = {20pt}, % 改行時インデント量 (20pt | Npt)
    % --- 行番号の設定 ---  
    numbers = {none}, % 行番号を左に表示 (none | left, right)
    numberstyle = {\scriptsize}, % 行番号の書式
    numbersep = {10pt}, % 行番号と本文の間隔 (10pt | Npt)
    stepnumber = {1}, % 行番号の増分(何行おきに行番号を表示するか)(1 | N)
    % --- Caption設定 ---
    captionpos = {t}, % Captionの位置 (t | t, b)
    % framexleftmargin = {10pt}, % 左マージン (0pt | Npt)
    % xleftmargin = {20pt}, % 左マージン (0pt | Npt)
}
\begin{document}
\section*{課題要旨}
本課題では,自販機のサーバプログラムを作成する.
複数クライアントからの接続に対して,\texttt{select()}を用いて,入出力を多重化する.
\section*{処理概要}
データ構造は,以下のとおりである.
クライアントデータ構造には,ソケットディスクリプタ,投入金額,現在購入しようとしている商品構造体のポインタを持つ.
クライアントの最大接続台数は5台とし,それぞれのクライアントデータ構造を配列で持つ.
商品データ構造体のポインタを持つことで,購入情報の冗長化を防ぐ.
\begin{center}
    \begin{minipage}[t]{.48\textwidth}
        \begin{lstlisting}[caption=商品構造]
typedef struct {
    char *name;
    int price;
    int stock;
} Product;
        \end{lstlisting}
    \end{minipage}
    \begin{minipage}[t]{.48\textwidth}
        \begin{lstlisting}[caption=クライアント構造]
typedef struct {
    int clSock;
    int coin;
    Product *drinkProduct;
} Client;
        \end{lstlisting}
    \end{minipage}
\end{center}
サーバは,自身のソケットを作成し,クライアントからの接続を待ち受ける.
ここで,\texttt{select()}を用いて,複数のクライアントからの接続を待ち受ける.
その際,デバッグ用として,標準入力を受け付けるため,\texttt{stdin}も監視対象に加える.
\paragraph{\texttt{stdin}からの入力があった場合}入力が\texttt{ls}ならば,在庫一覧を表示する.
また,入力が\texttt{cls}なら,接続中のクライアントリストを表示する.
EOFが入力された場合には,サーバを終了する.この時,接続中のクライアントにも終了を通知し,ソケットを全て閉じる.
\paragraph{クライアントからの接続があった場合}接続を受け付け,クライアントデータ構造を作成し,クライアントの構造体配列に追加する.\\
\textbf{\underline{クライアントが購入商品が未選択の場合}}
\begin{itemize}
    \item クライアントからの入力が\texttt{ls}の場合は,在庫一覧を送信する.
    \item クライアントからの入力がEOFの場合は,クライアントを切断し,クライアントデータ構造を削除する.
    \item クライアントからの入力が商品名の場合は,商品の在庫があるか確認し,在庫があれば,商品の構造体ポインタをクライアント構造体に設定する.
\end{itemize}
\textbf{\underline{クライアントが購入商品を選択している場合}}
\begin{itemize}
    \item クライアントからの入力がEOFの場合は,クライアントを切断し,クライアントデータ構造を削除する.
    \item クライアントからの入力が数値の場合は,投入金額を更新する.投入金額次第で,購入,返金,投入金額の表示を行う.
    \item 購入後は,商品の在庫を減らし,クライアントが保持している商品構造体ポインタをNULLにする.
\end{itemize}
\section*{工夫点}
\paragraph{グローバル変数の利用}グローバル変数の利用を最小限に抑えることで,意図せぬ変更を防ぐ.
グローバル変数として定義されているものは,\texttt{Product}構造体の配列,\texttt{Client}構造体の配列,プロダクトの個数である\texttt{int}型の変数である.
\paragraph{構造体のポインタの利用}構造体のポインタを利用することで,データの冗長化を防ぐ.
商品構造体のポインタをクライアント構造体に持たせることで,購入情報の冗長化を防ぐ.
\texttt{関数の分割}クライアントリスト,商品リストや名前から商品構造体を取得する関数など,関数を分割することで,可読性を向上させる.
グローバル変数を極力用いない設計にしたため,関数の引数にはしばしばポインタを用いる.
\section*{考察}
今回は,サーバからクライアントへの応答(商品リストなど)をテキストで行った.
しかし,データ形式をJSONなどにし,クライアント側で出力文を組み立てることで,ネットワークの負荷を軽減できると考える.
\section*{感想}
本課題,またこの講義を通して,現代のネットワークインフラを支える技術とその危険性,また我々開発者が心得るべきことを学んだ.
講義の内容は非常に興味深く,また実際にプログラムを書くことで,理解が深まった.
教科書にはない,現状の課題や技術についても深く教えてくださった敷田教授,ありがとうございました.
\begin{flushright}
    溝口 洸熙
\end{flushright}
\newpage
\section*{実行ファイルの生成と出力}
\begin{center}
    \begin{minipage}[t]{.48\textwidth}
        \begin{lstlisting}[caption=実行ファイルの生成]
$ make server
gcc -c server . c
gcc -c myio . c
gcc -o Server . out server . o myio . o
$ make client
        \end{lstlisting}
        \begin{lstlisting}[caption={サーバ側の出力}]
cls
>> Clients list :
| - Client [0]
| | - socket : 4
| | - ordered : none
| ‘- payed : 0
| - Client [1]
| | - socket : 5
| | - ordered : none
| ‘- payed : 0
ls
>> Products list to { fds : 1}
| - Product [ 0]: apple , price : 70 ,
left : 10
| - Product [ 1]: coffee , price : 100 ,
left : 1
| - Product [ 2]: milk , price : 40 ,
left : 8
| - Product [ 3]: orange , price : 80 ,
left : 12
| - Product [ 4]: tea , price : 50 ,
left : 15
        \end{lstlisting}
    \end{minipage}
    \begin{minipage}[t]{.48\textwidth}
        \begin{lstlisting}[caption={クライアント側出力例}]
$ nc 127.0.0.1 10000
|- Product[ 0]: apple  , price:  70, left: 10
|- Product[ 1]: coffee , price: 100, left:  1
|- Product[ 2]: milk   , price:  40, left:  8
|- Product[ 3]: orange , price:  80, left: 12
|- Product[ 4]: tea    , price:  50, left: 15
apple
The price is 70 yen
20
apple : 70 yen , you payed 20 yen .
50 ¥ more , please
60
apple : 70 yen , you payed 80 yen .
Here is your change : 10 yen
Enjoy your drink !
ls
|- Product[ 0]: apple  , price:  70, left:  9
|- Product[ 1]: coffee , price: 100, left:  1
|- Product[ 2]: milk   , price:  40, left:  8
|- Product[ 3]: orange , price:  80, left: 12
|- Product[ 4]: tea    , price:  50, left: 15
        \end{lstlisting}
    \end{minipage}
\end{center}
\end{document}