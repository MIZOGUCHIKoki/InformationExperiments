\documentclass[a4j,11pt]{jsarticle}
%jbbok:書籍・jsarticle:論文,短い文書・jreport:レポート・letter:手紙・twocolumn:2段組
\usepackage{amsmath,amsfonts,amssymb,mathtools,ascmac,bm,float,comment,url,fancybox,calc,subcaption,multicol,physics,tablefootnote,caption,graphics}
\usepackage{wrapfig}
\usepackage[T1]{fontenc}
\usepackage[margin=20truemm]{geometry}
\usepackage[dvipdfmx]{graphicx,color,hyperref}
\usepackage{tikz,listings,jlisting}
\usepackage{pxjahyper}
\usepackage{xcolor}
\hypersetup{
    colorlinks=true,
    citecolor=black,
    linkcolor=black,
    urlcolor=blue
}
        \tikzset{cy/.style={fill=cyan!10,rounded corners,rectangle,  text centered, minimum height=1cm,text width=4.1cm}}
        \tikzset{ye/.style={fill=yellow!10,rounded corners,rectangle,  text centered, minimum height=1cm,text width=4.1cm}}
        \tikzset{ma/.style={fill=magenta!10,rounded corners,rectangle,  text centered, minimum height=1cm,text width=4.1cm}}
        \tikzset{gr/.style={fill=green!10,rounded corners,rectangle,text centered,minimum height=1cm,text width=4.1cm}}
        \tikzset{Terminal/.style={rounded rectangle,  draw,  text centered, text width=3cm, minimum height=1.5cm}}
        \tikzset{Process/.style={rectangle,  draw,  text centered, text width=3cm, minimum height=1.5cm}}
        \tikzset{Decision/.style={diamond,  draw,  text centered, aspect=3,text width=5cm, minimum height=1.5cm}}
    \usetikzlibrary{arrows}
    \usetikzlibrary{intersections,calc,arrows.meta,backgrounds,shapes.geometric,shapes.misc,positioning,fit,graphs}
\renewcommand{\lstlistingname}{src.}
\lstset{
        %プログラム言語(複数の言語に対応,C,C++も可)
    language = Java,
        %背景色と透過度
    %backgroundcolor={\color[gray]{.90}},
        %枠外に行った時の自動改行
    breaklines = true,
        %自動改行後のインデント量(デフォルトでは20[pt])
    breakindent = 10pt,
        %標準の書体
    basicstyle = \ttfamily\small,
        %コメントの書体
    commentstyle = {\itshape \color[cmyk]{1,0.4,1,0}},
        %関数名等の色の設定
    classoffset = 0,
        %キーワード(int, ifなど)の書体
    keywordstyle = {\bfseries \color[cmyk]{0,1,0,0}},
        %表示する文字の書体
    stringstyle = {\ttfamily \color[rgb]{0,0,1}},
        %枠 tは上に線を記載, Tは上に二重線を記載
        %他オプション:leftline,topline,bottomline,lines,single,shadowbox
    frame = lines,
        %frameまでの間隔(行番号とプログラムの間)
    framesep = 5pt,
        %行番号の位置
    numbers = left,
        %行番号の間隔
    stepnumber = 1,
        %行番号の書体
    numberstyle = \small,
        %タブの大きさ
    tabsize = 4,
        %キャプションの場所(tbならば上下両方に記載)
    captionpos = t
}
    \setlength{\columnsep}{5mm}
    \columnseprule=0.1mm
    \renewcommand{\thefootnote}{*\arabic{footnote}}
    \renewcommand{\figurename}{Fig\ }
    \renewcommand{\tablename}{Tbl\ }
    \newcommand{\figref}[1]{Fig\ \ref{#1}}
    \newcommand{\tabref}[1]{Tbl\ \ref{#1}}
\makeatletter
\renewcommand{\thefigure}{%
\thesection.\arabic{figure}}
\@addtoreset{figure}{section}
\renewcommand{\thetable}{%
\thesection.\arabic{table}}
\@addtoreset{table}{section}
\makeatother


\title{\vspace{0cm}情報学群実験第1 最終レポート}
\author{1250373 溝口洸熙}
\date{\today}


\begin{document}
%\twocolumn[
\maketitle
%]
\begin{abstract}

\end{abstract}
\tableofcontents
\newpage
\section*{はじめに}
\addcontentsline{toc}{section}{はじめに}
\subsection*{レポートについて}
このレポートは,\LaTeXe を用いて作成している.また,図やグラフはTi\it{k}\normalfont zを用いて描画しており,ソースコードはlistingを用いている.
\subsection*{符号化と変数}
行列のステータスの符号と,新たに追加した変数を,\tabref{tbl:ini}に示す.
\begin{table}[h]
    \centering
    \caption{}
    \label{tbl:ini}
    \begin{minipage}[t]{0.3\linewidth}
        \centering
        \subcaption{符号とステータス}
        \label{tbl:符号とステータス}
        \begin{tabular}{cl}
            \hline
            符号 & \multicolumn{1}{c}{ステータス} \\
            \hline
            0    & 爆弾以外                       \\
            1    & 開かれたマス                   \\
            -1   & 爆弾                           \\
            -2   & フラグが立っている             \\
            \hline
        \end{tabular}
    \end{minipage}
    \begin{minipage}[t]{0.69\linewidth}
        \centering
        \subcaption{新たに追加した変数}
        \begin{tabular}{ll}
            \hline
            \multicolumn{1}{c}{変数名} & \multicolumn{1}{c}{役割}                               \\
            \hline
            \verb|int originalTable|   & \begin{tabular}{l}
                                                  生成した盤面の初期状態を記憶する.
                                              \end{tabular}                 \\
                                       &                                                        \\
            \verb|int tr|              & \begin{tabular}{l}
                                                             生成した盤面の初期状態を記憶する. \\
                                                             1手目で爆弾に当たることを回避するため.
                                                         \end{tabular} \\
            \hline
        \end{tabular}
    \end{minipage}
\end{table}
\newpage
\section*{仕様1}
\setcounter{section}{1}
\addcontentsline{toc}{section}{仕様1}
\begin{screen}
    \textbf{仕様1.}\\
    ゲーム開始時に,盤面にランダムに地雷を設置する.
\end{screen}
\subsection*{処理概要}
\begin{figure}[h]
    \centering
    \caption{仕様1の処理概要}
    \label{fig:仕様1の処理概要}
    \begin{tikzpicture}
        % \node[cy](initTable){\verb|void initTable()|};
        \node[Terminal](functionTable){\verb|initTable()|};
        \node[Terminal,below=0.5cm of functionTable](functionMine){\verb|setMine()|};
        \node[Decision,below=0.5cm of functionMine](checkBord){盤面を調査};
        \node[Process,below=0.5cm of checkBord](setNumber1){\verb|table|に\verb|0|を代入};
        \node[Process,below=0.5cm of setNumber1](setNumber2){\verb|originalTable|に\verb|0|を代入};
        \foreach \u \v in {functionTable/functionMine,functionMine/checkBord,setNumber1/setNumber2}
        \draw[-latex](\u)--(\v);
        \draw[-latex](checkBord)node[right,yshift=-40]{そのマスが爆弾でなければ}--(setNumber1);

        \node[Terminal,right =3cm of functionTable](setMine){\verb|setMine()|};
        \node[Process,below =0.5cm of setMine](set){\verb|numMine|の数だけ爆弾をセットする};
        \foreach \u \v in {functionMine/setMine,setMine/set,set/functionMine}
        \draw[-latex](\u)--(\v);
    \end{tikzpicture}
\end{figure}

\end{document}