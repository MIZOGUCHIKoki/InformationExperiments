\documentclass[a4j,11pt]{jsarticle}
%jbbok:書籍・jsarticle:論文,短い文書・jreport:レポート・letter:手紙・twocolumn:2段組
\usepackage{amsmath,amsfonts,amssymb,mathtools,ascmac,bm,float,comment,url,fancybox,calc,subcaption,multicol,physics,tablefootnote,caption,graphics}
\usepackage{wrapfig}
\usepackage[T1]{fontenc}
\usepackage[margin=20truemm]{geometry}
\usepackage[dvipdfmx]{graphicx,color,hyperref}
\usepackage{tikz,listings,jlisting}
\usepackage{pxjahyper}
\usepackage{xcolor}
\hypersetup{
    colorlinks=true,
    citecolor=black,
    linkcolor=black,
    urlcolor=blue
}
        \tikzset{cy/.style={fill=cyan!10,rounded corners,rectangle,  text centered, minimum height=1cm,text width=4.1cm}}
        \tikzset{ye/.style={fill=yellow!10,rounded corners,rectangle,  text centered, minimum height=1cm,text width=4.1cm}}
        \tikzset{ma/.style={fill=magenta!10,rounded corners,rectangle,  text centered, minimum height=1cm,text width=4.1cm}}
        \tikzset{gr/.style={fill=green!10,rounded corners,rectangle,text centered,minimum height=1cm,text width=4.1cm}}
        \tikzset{Terminal/.style={rounded rectangle,  draw,  text centered, text width=3cm, minimum height=1.5cm}}
        \tikzset{Process/.style={rectangle,  draw,  text centered, text width=3cm, minimum height=1.5cm}}
        \tikzset{Decision/.style={diamond,  draw,  text centered, aspect=3,text width=5cm, minimum height=1.5cm}}
    \usetikzlibrary{arrows}
    \usetikzlibrary{intersections,calc,arrows.meta,backgrounds,shapes.geometric,shapes.misc,positioning,fit,graphs}
    \renewcommand{\lstlistingname}{src.}
    \newcommand{\srcref}[1]{src. \ref{#1}}
    \AtBeginDocument{
        % \renewcommand{\thelstlisting}{\thesection.\arabic{lstlisting}}
    }
    \lstset{
            %プログラム言語(複数の言語に対応,C,C++も可)
        language = Java,
            %背景色と透過度
        %backgroundcolor={\color[gray]{.90}},
            %枠外に行った時の自動改行
        breaklines = true,
            %自動改行後のインデント量(デフォルトでは20[pt])
        breakindent = 10pt,
            %標準の書体
        basicstyle = \ttfamily\small,
            %コメントの書体
        commentstyle = {\ttfamily \color[cmyk]{1,0.4,1,0}},
            %関数名等の色の設定
        classoffset = 0,
            %キーワード(int, ifなど)の書体
        keywordstyle = {\bfseries \color[cmyk]{0,1,0,0}},
            %表示する文字の書体
        stringstyle = {\ttfamily \color[rgb]{0,0,1}},
            %枠 tは上に線を記載, Tは上に二重線を記載
            %他オプション:leftline,topline,bottomline,lines,single,shadowbox
        frame = leftline,
            %frameまでの間隔(行番号とプログラムの間)
        framesep = 5pt,
            %行番号の位置
        % numbers = left,
            %行番号の間隔
        stepnumber = 1,
            %行番号の書体
        numberstyle = \small,
            %タブの大きさ
        tabsize = 4,
            %キャプションの場所(tbならば上下両方に記載)
        captionpos = t
    }
    \setlength{\columnsep}{5mm}
    \columnseprule=0.1mm
    \renewcommand{\thefootnote}{*\arabic{footnote}}
    \renewcommand{\figurename}{Fig\ }
    \renewcommand{\tablename}{Tbl\ }
    \newcommand{\figref}[1]{Fig\ \ref{#1}}
    \newcommand{\tabref}[1]{Tbl\ \ref{#1}}
    \newcommand{\met}[1]{\ttfamily #1 \normalfont (\srcref{src:#1})の処理}
\makeatletter
\renewcommand{\thefigure}{%
\thesection.\arabic{figure}}
\@addtoreset{figure}{section}
\renewcommand{\thetable}{%
\thesection.\arabic{table}}
\@addtoreset{table}{section}
% \@addtoreset{lstlisting}{section}
\makeatother


\title{\vspace{0cm}情報学群実験第1 最終レポート}
\author{1250373 溝口洸熙\thanks{高知工科大学 情報学群 2年生}}
\date{\today}


\begin{document}
%\twocolumn[
\maketitle
%]

\begin{abstract}

\end{abstract}
\tableofcontents
\newpage
\section*{はじめに}
\addcontentsline{toc}{section}{はじめに}
\subsection*{レポートについて}
このレポートは,\LaTeXe を用いて作成している.図やグラフはTi\it{k}\normalfont Zを用いて描画しており,ソースコードはlistingを用いて表記している.
\subsection*{符号化と変数}
あるパネルのステータスを示す符号と,新たに追加したグローバル変数を,以下に示す.
\begin{table}[h]
    \centering
    \begin{minipage}[t]{0.3\linewidth}
        \centering
        \subcaption*{符号とステータス}
        \label{tbl:符号とステータス}
        \begin{tabular}{cl}
            \hline
            符号 & \multicolumn{1}{c}{ステータス} \\
            \hline
            0    & 爆弾以外                       \\
            1    & 開かれたパネル                 \\
            -1   & 爆弾                           \\
            -2   & フラグが立っている             \\
            \hline
        \end{tabular}
    \end{minipage}
    \begin{minipage}[t]{0.69\linewidth}
        \centering
        \subcaption*{新たに追加した変数}
        \begin{tabular}{ll}
            \hline
            \multicolumn{1}{c}{変数名} & \multicolumn{1}{c}{役割}                           \\
            \hline
            \verb|int originalTable|   & \begin{tabular}{l}
                                                  生成した盤面の初期状態を記憶する.
                                              \end{tabular}             \\
                                       &                                                    \\
            \verb|Boolean tr|          & \begin{tabular}{l}
                                                         2手目以降で\verb|true|になる変数. \\
                                                         1手目で爆弾に当たることを回避するため.
                                                     \end{tabular} \\
            \hline
        \end{tabular}
    \end{minipage}
\end{table}
\newpage
\section*{仕様1}
\setcounter{section}{1}
\addcontentsline{toc}{section}{仕様1}
\begin{screen}
    \textbf{仕様1.}\\
    ゲーム開始時に,盤面上へランダムに地雷を設置する.
\end{screen}
\subsection{処理概要}
\begin{figure}[h]
    \centering
    \caption{盤面上へランダムに地雷を設置する}
    \begin{tikzpicture}
        % \node[cy](initTable){\verb|void initTable()|};
        \node[Terminal](functionTable){\verb|initTable|};
        \node[Process,below=0.5cm of functionTable](functionMine){\verb|setMine|};
        \node[Decision,below=0.5cm of functionMine](checkBord){盤面を調査};
        \node[Process,below=0.5cm of checkBord](setNumber1){\verb|table|に\verb|0|を代入};
        \node[Process,below=0.5cm of setNumber1](setNumber2){\verb|originalTable|に\verb|0|を代入};
        \foreach \u \v in {functionTable/functionMine,functionMine/checkBord,setNumber1/setNumber2}
        \draw[-latex](\u)--(\v);
        \draw[-latex](checkBord)node[right,yshift=-40]{そのパネルが爆弾でなければ}--(setNumber1);

        \node[Terminal,right =3cm of functionTable](setMine){\verb|setMine|};
        \node[Process,below =0.5cm of setMine](set){\verb|numMine|の数だけ爆弾をセットする};

        \node[inner sep=0cm,fit={(setMine)(set)}](warp){};
        \node[inner sep=0cm,above=0.1cm of warp.north](caption){\verb|setMine|メソッド};
        \node[draw,dashed,rounded corners,fit={(warp)(caption)}](plate){};

        \foreach \u \v in {functionMine.east/plate,plate.west/functionMine}
        \draw[thick,-Stealth](\u)--(\v);
        \draw[-latex](setMine)--(set);
    \end{tikzpicture}
\end{figure}
\newpage
\subsection{処理}
\subsubsection*{\met{initTable}}
盤面を初期化するにあたって,以下の処理を行う.
\begin{enumerate}
    \renewcommand{\labelenumi}{\arabic{enumi})}
    \item 爆弾の設置パネルを\verb|setMine|メソッドで定める.
    \item \verb|table|の全ての行と列に\verb|0|を代入する.ただし,爆弾であるパネルは上書きしない.
          \begin{lstlisting}[caption=, label=, language=Java]
for (int x = 0; x < this.height; x++) {
    for (int y = 0; y < this.width; y++) {
        if (this.table[x][y] == -1) {// 爆弾がセットされている場所は避ける
            continue;
        }
        this.table[x][y] = 0; // 爆弾の場所以外は0で初期化
    }
}
    \end{lstlisting}
    \item \verb|originalTable|の全ての行に\verb|0|を代入する.ただし,爆弾であるパネルは上書きしない.
          \begin{lstlisting}[caption=, label=, language=Java]
this.originalTable[x][y] = 0;
    \end{lstlisting}
\end{enumerate}
\subsubsection*{\met{setMine}}
盤面に爆弾を配置するにあたって,以下の処理を行う.
\begin{enumerate}
    \renewcommand{\labelenumi}{\arabic{enumi})}
    \item 爆弾の個数を数える\verb|count|変数を定義する.
    \item 指定された爆弾の個数が\verb|count|になるまで,爆弾を配置する.
          \begin{lstlisting}[caption=, label=, language=Java]
while (count != this.numMine) {// numMineの数だけ爆弾をセットできたらループを抜ける
...
}
    \end{lstlisting}
    \item 爆弾の配置はランダムである.乱数で指定されたパネルが既に爆弾であれば再度乱数を生成し,爆弾が新たにセットできる場所では,\verb|table|,\verb|originalTable|の乱数値Indexを\verb|-1|に設定し,\verb|count|をインクリメントする.
          \begin{lstlisting}[caption=, label=, language=Java]
...
int x = new java.util.Random().nextInt(getHeight());
int y = new java.util.Random().nextInt(getWidth());
if (this.table[x][y] == -1) {
    // [x][y]にすでに爆弾がセットされていたら、もう一度乱数を決め直す
    continue;
}
count++;
...
    \end{lstlisting}
\end{enumerate}
\begin{flushright}
    \underline{仕様1 終}
\end{flushright}
\newpage
\section*{仕様2}
\setcounter{lstlisting}{0}
\setcounter{subsection}{0}
\addcontentsline{toc}{section}{仕様2}
\setcounter{section}{2}
\begin{screen}
    \textbf{処理2.}\\
    パネルを左クリックした際,クリックしたパネルを開く.
\end{screen}
\subsection{処理概要}
\begin{figure}[h]
    \centering
    \caption{タイルを開くときの処理}
    \begin{tikzpicture}
        \node[Terminal](openTile){\verb|openTile|};
        \node[Decision,below=0.4cm of openTile](turn){1手目で爆弾};
        \node[Decision,below=0cm of turn](flag){そのパネルにフラグが立っている};
        \node[Terminal,right=1cm of flag](return2){\verb|return;|};
        \node[Decision,below=0cm of flag](mine){爆弾を踏んだ};
        \node[Process,fill=magenta!10,below=0.4cm of mine](quarityMine){周辺の爆弾の個数を数えて表示};
        \node[Process,fill=magenta!10,below=0.4cm of quarityMine](setNumber){開いたパネルを\verb|-1|に設定};
        \node[Decision,below=0.4cm of setNumber](serch){全てのパネルが開いたか};
        \node[Terminal,below=0.4cm of serch](win){\verb|gui.win()|の呼び出し};

        \node[Process,right=1cm of turn](setMine){\verb|setMine|};
        \node[Terminal,right=1cm of mine](lose){\verb|gui.lose()|の呼び出し};
        \node[Terminal,right=1cm of serch](return){\verb|return;|};

        \foreach \u \v in {openTile/turn,mine/quarityMine,quarityMine/setNumber,setNumber/serch,serch/win,turn/setMine,mine/lose,serch/return,flag/return2}
        \draw[-latex](\u)--(\v);
        \draw[-latex](setMine)|-(openTile);
        \foreach \x \z in {turn/No,flag/No,mine/No,serch/Yes}
        \node[right=0.5cm of \x.south]{\z};
        \foreach \x \z in {turn/Yes,mine/Yes,serch/No/,flag/No}
        \node[above=0cm of \x.east]{\z};

        \node[inner sep=0.05cm,fit={(quarityMine)(setNumber)}](warp){};
        \node[dashed,rounded corners,draw,fit={(warp)}](plate){};
        \node[inner sep=0cm,left=0.1cm of plate.north west](caption){\textbf{仕様2}};

    \end{tikzpicture}
\end{figure}
\newpage
\subsection{処理}\label{sec:openTile}
\subsubsection*{\met{openTile}の一部}
\begin{enumerate}
    \renewcommand{\labelenumi}{\arabic{enumi})}
    %     \item 1手目で爆弾のタイルを開けてゲームオーバーではゲーム性が失われかねないので,1手目で爆弾である場合は,再度爆弾をセットし直し,もう一度同じタイルを開ける処理を行う.この際に仕様する変数が\verb|tr|であり,\verb|tr=false|の時は1手目,\verb|tr=true|の時は2手目以上であることを示している.
    %           \begin{lstlisting}[caption=, label=, language=Java]
    % if (this.table[x][y] == 1 && !this.tr) {// 1手目で爆弾ならば再度爆弾をセット
    %     this.setMine();
    %     this.openTile(x, y, gui);
    % }
    % \end{lstlisting}
    %     \item 開けたパネルに爆弾があれば,全てのタイルを表示後,"You lose"のポップアップを表示する.
    %           \begin{lstlisting}[caption=, label=, language=Java]
    % if (this.table[x][y] == -1) { // パネルに爆弾があった場合
    %     this.openAllTiles(gui);
    %     gui.lose();
    % }
    %     \end{lstlisting}
    %     \item もし,その場所にフラグが立っているのであれば,何もせずに\verb|return;|.この機能は追加した.
    %           \begin{lstlisting}[caption=, label=, language=Java]
    % else if (this.table[x][y] == -2) { // フラグがパネルに立っている場合
    %     return;
    % }
    %     \end{lstlisting}
    \item パネルに爆弾がない場合,その周辺の爆弾個数を\verb|returnMine|メソッドで取得し,そのパネルに表示する.
    \item 全てのパネルが開いたか否か確認する.
          \begin{lstlisting}[caption=, label=, language=Java]
int mineCount = this.returnMine(x, y, gui);// 周辺の爆弾の個数を調査
this.table[x][y] = 1; // 開かれたパネルの値を1に設定
...
Boolean jud = true;
for (int i = 0; i < getHeight(); i++) {
    for (int j = 0; j < getWidth(); j++) {
        if (this.table[i][j] == 0) {
            // 開いていないパネルがある場合judの値をtrueにする
            jud = true;
            break;
        } else { jud = false; }
    }
    if (jud) { break;}
}
if (!jud) { gui.win(); }		
\end{lstlisting}
\end{enumerate}
% \subsubsection{\met{returnMine}の一部}
\newpage
\section*{仕様3}
\addcontentsline{toc}{section}{仕様3}
\setcounter{section}{3}
\setcounter{subsection}{0}
\setcounter{lstlisting}{0}
\begin{screen}
    \textbf{仕様3.}\\
\end{screen}
\newpage
\section*{ソースコード}
\lstset{frame=shadowbox,numbers=left}
\addcontentsline{toc}{section}{ソースコード}
\begin{lstlisting}[caption=\ttfamily initTable, label=src:initTable, language=Java]
void initTable() {// 盤面を初期化する
  this.setMine();
    for (int x = 0; x < this.height; x++) {
      for (int y = 0; y < this.width; y++) {
          if (this.table[x][y] == -1) {// 爆弾がセットされている場所は避ける
            continue;
          }
          this.table[x][y] = 0; // 爆弾の場所以外は0で初期化
          this.originalTable[x][y] = 0;
      }
  }
}
\end{lstlisting}
\begin{lstlisting}[caption=\ttfamily setMine, label=src:setMine, language=Java]
void setMine() {// 爆弾をセット
  int count = 0;
  while (count != this.numMine) {
    // numMineの数だけ爆弾をセットできたらループを抜ける
    int x = new java.util.Random().nextInt(getHeight());
    int y = new java.util.Random().nextInt(getWidth());
    if (this.table[x][y] == -1) {
      // [x][y]にすでに爆弾がセットされていたら、もう一度乱数を決め直す
      continue;
    }
    this.table[x][y] = -1; // 爆弾の場所を値-1としてセットする
    this.originalTable[x][y] = -1;
    count++;
  }
}
\end{lstlisting}
\begin{lstlisting}[caption=\ttfamily openTile, label=src:openTile, language=Java]
public void openTile(int x, int y, MineSweeperGUI gui) {
    if (this.table[x][y] == 1 && !this.tr) {// 1手目で爆弾ならば再度爆弾をセット
        this.setMine();
        this.openTile(x, y, gui);
    }
    this.tr = true;
    if (this.table[x][y] == -1) { // パネルに爆弾があった場合
        this.openAllTiles(gui); gui.lose();
	} else if (this.table[x][y] == -2) { return; } // パネルにフラグが立っている場合
     else { // パネルに爆弾がなかった場合
			int mineCount = this.returnMine(x, y, gui);// 周辺の爆弾の個数を調査
			this.table[x][y] = 1; // 開かれたパネルの値を1に設定
			String mc = String.valueOf(mineCount);
			gui.setColorText(x, y, mineCount); gui.setTextToTile(x, y, mc); 
            Boolean jud = true;
			for (int i = 0; i < getHeight(); i++) {// 爆弾以外のパネルが全て開いているか確認
				for (int j = 0; j < getWidth(); j++) {
					if (this.table[i][j] == 0) {
						// 開いていないパネルがある場合judの値をtrueにする
						jud = true;
						break;
					} else { jud = false; }// 開いている場合falseにする
				}
				if (jud) { break; }
			}
			if (!jud) { gui.win(); }
		}
	}
\end{lstlisting}
\end{document}