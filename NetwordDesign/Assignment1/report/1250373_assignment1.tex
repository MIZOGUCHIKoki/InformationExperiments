\documentclass[12pt]{jlreq}
\usepackage{listings,xcolor,lastpage,framed}
\usepackage[left=10truemm,right=10truemm,top=20truemm,bottom=20truemm,headheight=22pt]{geometry}
\usepackage{fancyhdr}
\renewcommand{\texttt}[1]{{\ttfamily\bfseries\ #1\ }}
\fancyhf{}
\fancyhead[R]{\thepage\ / \pageref{LastPage}}
\fancyhead[C]{Assignment No.1}
\fancyhead[L]{1250373 溝口 洸熙}

\fancyfoot[R]{Compiled by Cloud LaTeX}
\fancyfoot[C]{Network-Design 2023}
\fancyfoot[L]{\fbox{Deadline: Jan. 8th, 2024 23:59}}
\pagestyle{fancy}
\lstset{ %ここで設定する内容は,全てのlstlisting環境で適用されます.
%【凡例】コマンド名 = {設定値} % コマンドの意味 (デフォルト値 | 選択肢)
  language = {C}, %利用言語
% --- コード書式設定 ---
  basicstyle = {\ttfamily\scriptsize}, % 基本書式
  commentstyle = {\color{green!50!black}\ttfamily\scriptsize}, % コメント書式
  keywordstyle = [1]{\color{red!50!black}\bfseries\ttfamily\scriptsize}, % キーワード書式:区分[1]
  keywordstyle = [2]{\color{red}\bfseries\ttfamily\scriptsize}, % キーワード書式:区分[2]
  keywordstyle = [3]{\color{orange}\bfseries\ttfamily\scriptsize}, % キーワード書式:区分[3]
  stringstyle = {\color{blue}\ttfamily}, % 文字列書式
  showspaces = {false}, % 空白表示 (false | true, false)
  showtabs = {false}, % Tabを表示 (false | true, false)
  tab = {}, % Tab記号( | $\mapsto$, $\to$など)
% --- その他の設定 ---
  frame = {single}, % フレーム設定 (none | leftline, topline, bottomline, lines, single, shadowbox, {T,t, B,b, L,l, R,r})
  framesep = {.1cm}, % フレームまでの間隔 ( 0cm | Npt, Ncmなど)
  breaklines = {true}, % 1行が長いとき,改行するか否かの設定 (false | true, false)
  breakindent = {20pt}, % 改行時インデント量 (20pt | Npt)
% --- 行番号の設定 ---  
  numbers = {left}, % 行番号を左に表示 (none | left, right)
  numberstyle = {\scriptsize}, % 行番号の書式
  numbersep = {10pt}, % 行番号と本文の間隔 (10pt | Npt)
  stepnumber = {1}, % 行番号の増分(何行おきに行番号を表示するか)(1 | N)
% --- Caption設定 ---
  captionpos = {t} % Captionの位置 (t | t, b)
}
\begin{document}
\section*{課題要旨}
低レベルAPIである\texttt{open},\texttt{close},\texttt{read}は,それぞれファイルを開く,閉じる,読み込むという機能を持つ.
バッファリングは,ファイルの読み書きを効率化するために行われるが,低レベルAPIでは行われない.
本課題ではバッファリングをする高レベルAPIを作成する.

\section*{ソースコードの説明}
ソースコードは別途提出しているほか,付録としてp.\pageref{apendix}に貼付している.
\paragraph{構造体の定義}
バッファリングをするに際して,「ファイル」を定義するため,その構造体内にバッファを持つ必要がある.
バッファは\texttt{buffer}として\texttt{char}型の配列で定義し,配列の位置を指すポインタを\texttt{index}として\texttt{int}型で定義する.
また,バッファサイズをヘッダで\texttt{MyBufferSize}として定義する.
\texttt{int}型の\texttt{count}は,\texttt{read}を用いて\texttt{buffer}へ何オクテット書き込んだかを保持する.
この構造体を\texttt{my\_file}として定義する.

\paragraph{ファイルを開く}
\texttt{my\_file}型のポインタ関数として\texttt{my\_fopen}を定義する.
ファイル名のポインタを引数として受け取る.
\texttt{open}を用いてファイルを開き,そのファイルディスクリプタを変数\texttt{fd}に保持する.
正常にファイルが開かれた場合,\texttt{my\_file}型のポインタ変数\texttt{*fp}を宣言し,そのために必要なメモリを\texttt{malloc}を用いて動的に確保する.
それぞれのメンバを初期化\footnote{本課題では,学習のためソースコード上ではアロー演算子を利用していない.\texttt{(*fp).index = 0}と\texttt{fp->index = 0}は同義である.}して,\texttt{fp}を返す.
正常にファイルを開けなかった場合は,\texttt{NULL}を返す.

\paragraph{ファイルを閉じる}
ファイルを閉じる関数として\texttt{int}型の\texttt{my\_fclose}を定義する.
引数として,\texttt{my\_file}型のポインタを受け取る.
\texttt{close}を用いてファイルを閉じる.\texttt{close}の仕様により,正常に閉じられた場合は\texttt{0}を,閉じられなかった場合は\texttt{-1}を返す.
このとき,\texttt{my\_fopen}で動的に確保したメモリを\texttt{free}を用いて解放する.

\paragraph{1文字単位の入力}
指定されたファイルを1文字単位で入力する関数として\texttt{int}型の\texttt{my\_getc}を定義する.
引数として,\texttt{my\_file}型のポインタを受け取る.
バッファリングの機構として,\texttt{fp->index}が\texttt{fp->count}より大きくなった場合,つまりバッファを全て読み取った場合,または\texttt{fp->buffer}が空である場合は,\texttt{read}を用いて\texttt{MyBufferSize}オクテットを読み取り,\texttt{fp->buffer}に格納する.
このとき,\texttt{fp->index},を初期化し,\texttt{fp->count}に\texttt{read}で読み取ったオクテット数を格納する.
ここで,\texttt{read}で読み取ったオクテッド数が0以下である場合,\texttt{EOF}を返す.
\texttt{size =  0}のとき,これ以上読み取ることができない状況であり,\texttt{size = -1}のとき,エラーが発生した状況である.
\texttt{fp->buffer}に内容があり,\texttt{(*fp).index}が読み取ったサイズと等しくなるまで,\texttt{fp->buffer}上の\texttt{fp->index}番地を返し,\texttt{fp->index}をインクリメントする.
\texttt{fp->buffer}がすべて読み取られた場合,\texttt{EOF}を返す.

\section*{考察}
バッファリングの機構により,\texttt{read}を用いてファイルを読み取る回数が減り,ファイルの読み取りが効率化されるとされている.
しかし,配列が空であるかどうかを判定するために,今回であれば\texttt{MyBufferSize}回の判定を行う必要がある.
このため,\texttt{MyBufferSize}が大きいほど,初回のファイル読み取りに時間がかかると考えられる.
グローバル関数に\texttt{is\_buffer\_empty}を定義し,\texttt{fp->buffer}が空であるかどうかをファイルを開いて初めて\texttt{my\_fgetc}を呼び出すときに判定することで,\texttt{MyBufferSize}回の判定を行う必要がなくなる.

\section*{感想}
すでにある\texttt{read},\texttt{open}などの仕様をしっかりと確認しつつ,コーディングすることが求められた.
本来,プログラミングする際に,用いる関数の仕様をしっかりと確認することが重要であるが,今回の課題を通して,その重要性を再認識することができた.
\newpage
\section*{付録:ソースコード}
\label{apendix}
\lstinputlisting[caption={\texttt{assignment1.c}}]{../assignment1.c}
\lstinputlisting[caption={\texttt{output}}]{../out.txt}
\end{document}