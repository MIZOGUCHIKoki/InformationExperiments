\documentclass[a4j,11pt]{jsarticle}
%jbbok:書籍・jsarticle:論文,短い文書・jreport:レポート・letter:手紙・twocolumn:2段組
\usepackage{amsmath,amsfonts,amssymb,mathtools,ascmac,bm,float,comment,url,fancybox,calc}
\usepackage[T1]{fontenc}
\usepackage[margin=20truemm]{geometry}
\usepackage[dvipdfmx]{graphicx}
\usepackage{tikz,listings,jlisting}
\renewcommand{\lstlistingname}{src.}
\lstset{
        %プログラム言語(複数の言語に対応,C,C++も可)
    language = Java,
        %背景色と透過度
    %backgroundcolor={\color[gray]{.90}},
        %枠外に行った時の自動改行
    breaklines = true,
        %自動改行後のインデント量(デフォルトでは20[pt])
    breakindent = 10pt,
        %標準の書体
    basicstyle = \ttfamily\small,
        %コメントの書体
    commentstyle = {\itshape \color[cmyk]{1,0.4,1,0}},
        %関数名等の色の設定
    classoffset = 0,
        %キーワード(int, ifなど)の書体
    keywordstyle = {\bfseries \color[cmyk]{0,1,0,0}},
        %表示する文字の書体
    stringstyle = {\ttfamily \color[rgb]{0,0,1}},
        %枠 tは上に線を記載, Tは上に二重線を記載
        %他オプション:leftline,topline,bottomline,lines,single,shadowbox
    frame = lines,
        %frameまでの間隔(行番号とプログラムの間)
    framesep = 5pt,
        %行番号の位置
    %numbers = left,
        %行番号の間隔
    stepnumber = 1,
        %行番号の書体
    numberstyle = \small,
        %タブの大きさ
    tabsize = 4,
        %キャプションの場所(tbならば上下両方に記載)
    captionpos = t
}
    \usetikzlibrary{arrows}
    \usetikzlibrary{intersections,calc,arrows.meta,backgrounds,shapes.geometric,shapes.misc,positioning,fit}

    \setlength{\columnsep}{5mm}
    \columnseprule=0.1mm
    \renewcommand{\thefootnote}{\arabic{footnote})}

\newcounter{point}
\setcounter{point}{0}
\newcommand{\Point}[1][]{\refstepcounter{point}{#1}工夫点{\thepoint}}
                   
\title{\vspace{0cm}Study課題 02}
\author{1250373 溝口洸熙\thanks{高知工科大学 情報学群 学士2年}} 
\date{\today}

\begin{document}
%\twocolumn[
\maketitle
%]
\begin{abstract}
    このレポートは,Study課題の各問題の工夫点をまとめたものである.コードの転記には\verb|listings|,\verb|jlisting|を用いており,描画には\ Ti\emph{k}\normalfont Z\ を用いている.このレポートは,ソースコードの行番号を消している.
\end{abstract}
\section*{\Point}
\noindent\underline{汎用性の高いシステムの構築}\par
今回は,自動販売機に登録できる商品の個数が$5$個と決まっていたが,実際の自動販売機は,自動販売機の機種によって登録できる商品の個数が変わる.
プログラム上で,定数を用いて最大の商品登録個数を決めると,別の機種への適用が難しくなる.ここで,変数を用いてその変数をコードに適用することで,簡単に最大登録個数を変更できる仕組みを作成した.(src.\ref{最大登録個数})
\begin{lstlisting}[caption=最大登録個数の柔軟化, label=最大登録個数, language=Java]
int index_length= 5;// 最大登録個数 
// --一部抜粋--
for (int i = 0; i < index_length; i++) {
 if (data_drink[i] == null) {
   System.out.println(i + 1 + "番 -----未登録-----");
\end{lstlisting}
\ \ また,最大登録個数を変数を用いて保存することで,購入する際の「登録されていない」という出力判定をより簡単にした.今回の場合$6$以上の数を入力した場合,登録されていないと出力される.

\newpage
\section*{\Point[\label{金銭授受システム}]}
\noindent\underline{実際の自販機に寄せた金銭授受システム}\par
実際の自動販売機は,$1$円玉,$5$円玉は処理されない.自販機は硬貨の取り扱いのコストが,かかることがあるので,$1$円単位の取引はしないことがほとんどである.そこで,$1$円以上$10$円未満のお金は自動的に出力する仕組みを実装した.(src.\ref{金銭授受コード})
\begin{lstlisting}[caption=$1$円単位のお金を出力するシステム, label=金銭授受コード, language=Java]
int insert_money = new java.util.Scanner(System.in).nextInt();
if (insert_money % 10 != 0) {
    System.out.println(insert_money % 10 + "円が出力");
    insert_money = insert_money - insert_money % 10;
}
\end{lstlisting}
\section*{\Point{}}
\noindent\underline{商品登録時の金額修正}\par
工夫点\ref{金銭授受システム}で述べた理由より,商品価格は$10$円単位で行う必要がある.もし仮に何らかのミスで$1$円単位の価格設定になった場合,そのミスを指摘しプログラムを終了する仕組みを実装した.(sec.\ref{金額の是正})
\begin{lstlisting}[caption=設定金額の検証, label=金額の是正, language=Java]
public Drink(String name, int price) {// コンストラクタ
  this.name = name;
  if (price % 10 != 0) {
    System.out.println(this.name + "の金額を,10円単位で設定してください.");
    System.exit(0);
  } else {
    this.price = price;
  }
  this.name = name;
}
\end{lstlisting}
仮に,ソーダを$138$円で登録した場合の処理を記す.
\begin{itembox}[l]{実行結果}
\verb|$ java Study02 |\ovalbox{Enter}
\\
\verb|ソーダの金額を,10円単位で設定してください|
\end{itembox}
\end{document}