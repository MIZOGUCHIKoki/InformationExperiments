\documentclass[a4j,11pt]{jsarticle}
%jbbok:書籍・jsarticle:論文,短い文書・jreport:レポート・letter:手紙・twocolumn:2段組
\usepackage{amsmath,amsfonts,amssymb,mathtools,ascmac,bm,float,comment,url,fancybox,calc,subcaption}
\usepackage{wrapfig}
\usepackage[T1]{fontenc}
\usepackage[margin=20truemm]{geometry}
\usepackage[dvipdfmx]{graphicx,color,hyperref}
\usepackage{tikz,listings,jlisting}
\usepackage{pxjahyper}
\usepackage{xcolor}
\hypersetup{
    colorlinks=true,
    citecolor=black,
    linkcolor=black,
    urlcolor=blue
}

    \usetikzlibrary{arrows}
    \usetikzlibrary{intersections,calc,arrows.meta,backgrounds,shapes.geometric,shapes.misc,positioning,fit}

    \setlength{\columnsep}{5mm}
    \columnseprule=0.1mm
    \renewcommand{\thefootnote}{\arabic{footnote})}
    \captionsetup[subfigure]{labelformat=simple}
    \renewcommand{\thesubfigure}{(\alph{subfigure})}

\newcounter{point}
\setcounter{point}{0}
\newcommand{\Po}[1][]{\refstepcounter{point}変更点{\thepoint}}
    \title{\vspace{0cm}Study課題 05}
    \author{1250373 溝口洸熙\thanks{高知工科大学 情報学群 学士2年}} 
    \date{\today}
\renewcommand{\lstlistingname}{src.}
\lstset{
        %プログラム言語(複数の言語に対応,C,C++も可)
    language = Java,
        %背景色と透過度
    %backgroundcolor={\color[gray]{.90}},
        %枠外に行った時の自動改行
    breaklines = true,
        %自動改行後のインデント量(デフォルトでは20[pt])
    breakindent = 10pt,
        %標準の書体
    basicstyle = \ttfamily\small,
        %コメントの書体
    commentstyle = {\itshape \color[cmyk]{1,0.4,1,0}},
        %関数名等の色の設定
    classoffset = 0,
        %キーワード(int, ifなど)の書体
    keywordstyle = {\bfseries \color[cmyk]{0,1,0,0}},
        %表示する文字の書体
    stringstyle = {\ttfamily \color[rgb]{0,0,1}},
        %枠 tは上に線を記載, Tは上に二重線を記載
        %他オプション:leftline,topline,bottomline,lines,single,shadowbox
    frame = lines,
        %frameまでの間隔(行番号とプログラムの間)
    framesep = 5pt,
        %行番号の位置
    numbers = left,
        %行番号の間隔
    stepnumber = 1,
        %行番号の書体
    numberstyle = \small,
        %タブの大きさ
    tabsize = 4,
        %キャプションの場所(tbならば上下両方に記載)
    captionpos = t
}

\begin{document}
%\twocolumn[
\maketitle
%]
\begin{abstract}
    このレポートは,Study課題の変更点をまとめたものである.コードの転記には\verb|listings|,\verb|jlisting|を用いており,描画には\ Ti\emph{k}\normalfont Z\ を用いている.このレポートは,ソースコードの行番号を消している.「変更点$n$」はカウンタを用いている.また,対戦相手が人間でない場合は,NPC(Non Player Character)と表記する.
\end{abstract}
\section*{前提}
配布されたファイル\verb|Board.java|内にあったメソッドを全て使用している.
\section*{\Po:先手後手の判定}
このボードゲームの変更前は,人間とNPCとの対戦であり,人間が先手で固定されていた.変更後は,NPCとNPCとの対戦であり,先手か後手かの判別を記録するシステムはやや分かりにくい.\par
そこで,プレイヤーが2人存在することを考えて,変数\verb|turn|を設け,何度碁盤が操作されたかを記録する.\verb|turn|を2で割った剰余を判定材料とし,先手か後手かの判断をしている.
\section*{\Po:コメントの工夫}
メソッドの返り値が\verb|Boolean|型の場合,\verb|true,false|のそれぞれの場合どのような意味を成すのか,コメントすることを心掛けた.
たとえば,\verb|Board.java|の\ovalbox{\verb|isEnd();|}メソッドにおいて,\verb|true|の場合は「空白あり」,\verb|false|の場合は「空白なし」のコメントを付している.
\section*{\Po:重複処理の削除}
NPC同士の対戦であるので,双方の操作が同一である.つまり,先手の処理と後手の処理を人間 対 NPCの時のように分ける必要がないので,
盤面の表示,勝利判定などは1度でよく,非常にシンプルなコードになった.(src.\ref{code})
\newpage
\begin{lstlisting}[caption=\ttfamily C4.java, label=code, language=Java]
while (true) {
  int x = -1, y = -1;
  turn++;
  do {
    x = rand.nextInt(5);
    y = rand.nextInt(5);
  } while (b.isLegal(x, y));
  if(turn % 2 == 0){
    System.out.println("先手は " + x + " " + y + " に置きました");
  }else{
    System.out.println("後手は " + x + " " + y + " に置きました");
  }

  b.showBoard();// 盤面表示

  if(b.isWinning()) {// 勝利判定
    break;
  }

  if(!b.isEnd()){// 空白の有無判定
    hasEmpty = false;
    break;
  }
}
\end{lstlisting}
\end{document}
