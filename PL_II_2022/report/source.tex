\renewcommand{\thesection}{\Alph{section}}
\chapter*{付録}
\addcontentsline{toc}{chapter}{付録}
\setcounter{section}{0}
\section{ソースコード}
\renewcommand{\thelstlisting}{src \thesection.\arabic{lstlisting}}
\renewcommand{\thetable}{Tbl \thesection.\arabic{table}}
\begin{lstlisting}[caption={{\ttfamily sort.s}},label={src:sort.s},frame={shadowbox}]
    section .text
    global  sort
  sort:
    push  esi
    push  edi
    push  edx
    push  ecx
    push  ebx
    push  eax
    dec   ecx
  loop0:
    cmp   ecx,  0 ; ecx = i
    jle   endp
    mov   edx,  [ebx]   ; max = data[0]
    mov   eax,  0       ; max_indent = 0
    mov   edi,  1 ; edi = j
    loop1:
      cmp   edi,  ecx   ; j > i?
      jg    loop0l
      mov   esi,  [ebx + edi*4] ; data[j]
      cmp   esi,  edx       ; data[j] >= max
      jge   then
      jmp   endif
      then:
        mov edx,  [ebx + edi*4] ; max = data[j]
        mov eax,  edi           ; max_index = j
      endif:
        inc edi
        jmp loop1
    loop0l:
      mov   esi,  [ebx + eax*4]   ; m = data[max_index]
      mov   edi,  [ebx + ecx*4]   ; edi = data[i]
      mov   [ebx + eax*4],  edi   ; data[max_index],  data[i]
      mov   [ebx + ecx*4],  esi   ; data[i] = m
      dec   ecx
      jmp   loop0
  endp:
    pop eax
    pop ebx
    pop ecx
    pop edx
    pop edi
    pop esi
    ret
\end{lstlisting}
\begin{lstlisting}[caption={{\ttfamily test\_sort.java}},label={src:testsort.s},frame={shadowbox}]
  section .text
  global  _start
  extern  sort, print_eax

_start:
  mov ebx,  data
  mov ecx,  ndata
  call  sort      ; ソート
  mov edi,  0      
loop:   ; 結果の出力
  cmp edi,  ndata
  je  endp
  mov eax,  [data + edi * 4]
  call print_eax
  inc edi
  jmp loop


endp:
  mov eax,  1
  mov ebx,  0
  int 0x80

  section .data
data:  dd 1, 3, 5, 7, 9, 2, 4, 6, 8, 0, 1, 2
ndata  equ ($ - data) / 4  
\end{lstlisting}
\begin{lstlisting}[language={Java},caption={{\ttfamily sort.java}},label={src:sort.java},frame={shadowbox}]
class sort{
    public static void main(String[] args){
        int[] data = {1, 3, 5, 7, 9, 2, 4, 6, 8, 0, 1, 2};
        buble(data);
        for(int d: data){
            System.out.println(d);// 出力
        }
    }
    private static void buble(int[] data) {
        int max_index = 0; int max = 0;
        for (int i = data.length - 1; i > 0; i--) {
            max = data[0]; max_index = 0;
            for (int j = 1; j <= i; j++) {
                if (data[j] >= max) { max = data[j]; max_index = j; }
            }
            int m = data[max_index]; data[max_index] = data[i]; data[i] = m;
        }
    }
}
\end{lstlisting}