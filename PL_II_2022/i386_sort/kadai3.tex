\chapter{低級言語での理論的計算量に関する実験}\label{chap:時間計測}
\section{実験の目的}
今回の実験の目的は,アセンブリ言語・機械語で直接記述した場合も実行時間は論理的計算量(アルゴリズムによって\(O(n^2)\)や\(O(n\log n)\)など)に従うことを示すことである.選択ソートのアルゴリズムの計算量は\(O(n^2)\)である\cite[p.50,51]{アルゴリズムとデータ構造}ので,アセンブリ言語で記述した選択ソートアルゴリズムもそれに従うか検証するれば,目的を満たすことになる.
\section{実験方法}\label{sec:時間計測}
より正確に選択ソートの実行時間を計測するため,\testsort ファイルを\print を利用する箇所を削除し,\ref{src:データの個数指定}に書き換える.\par
\begin{wrapfigure}{r}[1pt]{0.3\textwidth}
   \vspace{-3em}
   \begin{lstlisting}[language={Bash},caption={時間計測実行コマンド},label={src:command3},frame={single},numbers={none}]
$ time ./a.out
-- 実行結果(略)--
real    0m0.002s
user    0m0.001s
sys     0m0.000s
\end{lstlisting}
   \vspace{-3em}
\end{wrapfigure}
実験環境は,\ref{usingPC}に示した通り.アセンブリ言語での実行時間の検証は,Linux標準の{\ttfamily time}コマンドを用いて計測する.\ref{command1}のコマンドを実行し,実行ファイル{\ttfamily a.out}を生成した後,\ref{src:command3}を実行し実行時間を計測する.
各実行時間の中でも{\ttfamily real}が引数コマンドを実行するのにかかった時間である.実験回数は1つのテストにつき3回行い実行時間を平均する.
さらに,選択ソートは最良時間計算量と最悪時間計算量が等しく\cite[p.50]{アルゴリズムとデータ構造},整列対象のデータ列は計測時間に依存しない故,今回はデータ列を全て{\ttfamily 0}に定める.
実験結果をExcelを使って多項式近似(2次)を求め,その関数曲線と実験結果を比較する.テストデータ数は\eqref{equ:テストデータ}の集合\(T\)である.\\
\begin{minipage}[c]{0.43\textwidth}
   \centering
   \hspace*{2em}
   \begin{lstlisting}[caption={\testsort 書き換え後}, label={src:データの個数指定},frame={single},numbers={none}]
略
   extern  sort
_start:
mov ebx,    data
mov ecx,    ndata
call sort
mov eax,    1
mov ebx,    0
int 0x80
   section .data
data: times データ個数 dd 0
ndata: equ ($ - data) / 4
   \end{lstlisting}
\end{minipage}
\begin{minipage}[c]{0.49\textwidth}
   \centering
   \begin{align}
       T_1 & =\{10^n\mid n\in\mathbb{N}, n\leq 5\} \notag \\
       T_2 & =\{n\times 10^4\mid 2\leq n\leq 9\}  \notag  \\
       T   & =T_1\cup T_2\label{equ:テストデータ}
   \end{align}
   \hspace{1em}
\end{minipage}
\section{実験結果}
実験結果\ref{fig:データの個数と実行時間の曲線グラフ}に示す.実行時間実験結果詳細は,\ref{tblr:実行時間計測実験結果A}に掲載している.
\begin{figure}[htb]
   \centering
   \caption{アセンブリ言語のデータの個数と実行時間の曲線グラフ}
   \label{fig:データの個数と実行時間の曲線グラフ}
   \begin{tikzpicture}
       \begin{axis}[
               xlabel={データ個数(個)\(n\)},
               ylabel={実行時間(秒)\(t\)},
               enlarge x limits=false,
               width=0.9\textwidth,
               height=0.3\textheight,
               ytick distance=0.5,
               ymin=0,
               legend = inner,
               legend pos=north west,
           ]
           \addplot[black,thick,mark=*] table[x=n, y=av, col sep=comma]{exdataA_all.csv};
           \addlegendentry{実験結果}
           \addplot[dashed,domain=0:100000,thin,mark=none,samples={300}] plot(\x, {6 * pow(10,-10) * (\x)^2 + 9 * pow(10,-7) * (\x) - 0.0004});
           \addlegendentry{\(t=6\cdot 10^{-10}n^2+9\cdot 10^{-7}n-0.0004\)}
       \end{axis}
   \end{tikzpicture}
\end{figure}
\begin{align}
   \intertext{実行時間を\(t\),データの個数を\(n\)とすると,近似多項式は\eqref{equ:近似多項式}.}
   t= 6\cdot 10^{-10}n^2+9\cdot 10^{-7}n-0.0004\label{equ:近似多項式}
\end{align}
\section{考察}
オーダ記法は,アルゴリズムの時間計算量の入力サイズ\(n\)を用いた関数\(f(n)\)に対して,その関数の主要項の係数を削除した\(O(f(n))\)である.\cite[p.7]{アルゴリズムとデータ構造}\par
従って,\eqref{equ:近似多項式}を元にオーダ記法の定義より,このアルゴリズムの計算量は\(O(n^2)\)であることが分かる.
